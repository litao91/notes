\documentclass[11pt]{book}
\usepackage{parskip}
\usepackage{spverbatim}
\usepackage[top=1.8cm, bottom=2.5cm, left=1.5cm, right=1.5cm]{geometry}
\usepackage{fontspec}
\setmainfont{DejaVu Serif}
\begin{document}
\chapter{Personalizing Vim}
\begin{description}
\item[Changing the font] \verb|set guifont=Courier\ 14| Courier can be exchanged
with the name of any font 14 with any font size.
\item[Changing Color Scheme] \verb|colorscheme mycolor|
\item[personal Highlighting] \verb|match Group /pattern/| Two arguments, the
first one is the name of the color group, the second is the actual pattern you
want to match. use \verb|match NONE| to cancel match, also we have \verb|2match|
\verb|3match| to have second and third match.
\item[Defining color group]
 \verb|hightlight MyGroup ctermbg=red guibg=red gctermfg=yellow guifg=yellow term=bold|
\item[Set how the status line should look] \verb|set statusline format|, format
is a string such as printf that describes how the status line should look. Use
\verb|help stausline| to find the detail. Example
\begin{spverbatim}
set statusline=%F%m%r%h%w\ [FORMAT=%{&ff}]\ [TYPE=%Y]\ [ASCII=\%03.3b]\ [HEX=\%02.2B]\ [POS=%04l,%04v]\ [%p%%]\ [LEN=%L]
\end{spverbatim}
 \item[Make sure status line always shown] \verb|set laststatus=2|
\item[Remove menus in Gvim] \verb|set guioptions-=m|, \verb|set guioptions-=T|, 
see \verb|help 'guioptions'|
\item[Modifying tabs]\verb|set tabline=tabline-layout| and \verb|set guitablabel|
 for gui. It's very similar to one used for the status line.
\item[Remove tab] \verb|set showtabline=0|
\item[Set tab tip] \verb|guitabtooltip|, the value you want to show when the
mouse cursor hovers the tab.
\item[Cursor tracking] \verb|set cursorline|, we can add style
 \verb|highlight CursorLine guibg=lightblue ctermbg=lightgray|, For column:
\verb|set cursorcolumn|
\item[Adding Line number] \verb|set number|, change the default number of
columns used for the line numbers:\verb|set numberwidth=XXX|, XXX is number
\item[Spell Checking]\verb|set spell| and \verb|set spelllang=en,de,da| to set
language. Pressing Z + = to have a list of good guesses.
 \verb|set spellsuggest=X| to set number of alternative ways of spelling
\item[ballons] In Vim, tool tips for the editing area are called
\textbf{ballons} and they are only shown when the cursor is hovering over one of
the characters. The commands you will need to know in order to use the ballon
are
    \begin{itemize}
        \item Turn on \verb|set ballooneval|
        \item Howlong it should wait before showing \verb|set balloondelay=400|
        \item Sets the string that vim will show in the ballon:
            \verb|set balloonexpr="textstring"| can be a static string or return
            from some function
    \end{itemize}
\item[Abbreviations] In vim, abbreviations are created withn:
    \begin{itemize}
        \item \verb|:abbreviate|: abbreviations for all modes
        \item \verb|:labbrev|: abbreviations for insert mode
        \item \verb|:cabbrev|: abbreviations for the command mode only.
    \end{itemize}
    All of the commands take two arguments - the abbreviation and the full text
    should expand to.
    Simply place a file called abbrevation.vim and put in your VIMHOME, and
    simple call \verb|source: $VIM/abbreviation.vim| in your vimrc
\item[Modifying key bindings]
    \begin{itemize}
    \item \verb|:map| for all mode
    \item \verb|:imap| for insert mode only
    \item \verb|:cmap| for command-line mode only
    \item \verb|:nmap| for the Normal mode only
    \item \verb|:vmap| for visual mode only
    \end{itemize}
    e.g Ctrl+S to save:\verb|:map <C-s> :w<cr>| the \verb|<cr>| to execute the
    command. Check \verb|help key-mapping|
\end{description}

 \chapter{Better Navigation}
 \section{Context aware navigation:}
 \begin{itemize}
 \item \verb|{| Go to gthe beginning of the paragraph, or in the empthy line
  just above it, \verb|}| end of the paragraph
 \item \verb|g,| go the place where you make change previously, using this
 comman dseveral times in a row will loop you through locations of previous
 changes in the file.\verb|g;| will let you move forward.
 \item \verb|(:| Move to the beginning of the sentence and \verb|):| end of the
 sentence.
 \item \verb|w| Move to the beginning of the next word.
 \item \verb|b| Move to the beginning of the previous word.
 \item \verb|e| Move to the end of the world.
 \item \verb|ge| Backward to the end of word.
 \item \verb|%| Go to the other side of the block.
 \item \verb|(|/\verb|)| sentences backward/forward.
 \item \verb|{|/\verb|}| paragraphs backward/forward
 \item \verb|]]| sections forward or to the next '{' in the first column.
 \item \verb|][| section forward or to the next '}' in the first column.
 \item \verb|[[| sections backward or to the previous '{' in the first column
 \item \verb|[]| sections backward or to the previous '}' in the first column
 \item \verb|[{|Move to the beginning of the block, \verb|]}| move to the end of
 the blcok
 \item \verb|[/| Move to the beginning of the comment block, and \verb|]\| for
 the end.
 \item \verb|gd| go to declaration , start search from current line. The
 \verb|gD| start from line one.
 \item \verb|Ctrl+]| to follow the link
 \end{itemize}

 Faster navigation in multiple buffers:
 For every file you open, you have a Vim buffer.
 \begin{itemize}
 \item \verb|:buffers| Bring up the list of buffers and find the right buffer in
 the list. Note that this is not interactive.
 \item With the number N of the buffer where the file is place, you can switch
 to that with: \verb|:buffer N|.
 \item Go cycle through: \verb|bnext| \verb|bprevious|
 \item Open referenced files faster: \verb|gf|
 \end{itemize}

 \section{Search and you will find}
 Search In current file:
 \begin{itemize}
 \item \verb|?someWord| the command searches backwards in the file for the first
 word after the question mark.
 \item \verb|/someWord| Search forward.
 \item \verb|g#| and \verb|g*| Now vim not just jumps to the next occurrence of
 the word, but also to any occurrences where the world is part of another word.
 \end{itemize}
 Search in multiple files:
 \begin{itemize}
 \item \verb|:vimgrep /pattern/[j][g] file fiel2 ... fileN|, this command takes
 two arguments. The first is the pattern you want to search for. You can use
 Vim's regular expressions in the pattern or you can just write a word. You can
 add either the two flags j and g. The flags help you select how much to get
 your result, and how it should presented to you. If the g flag is added, then
 the result will include a line for each match of the pattern. If the j flag is
 added to the end of your pattern, then you will not be presented with the
 result. I will just be updated into your quickfix list for later retrieval. The
 second argument to vimgrep is the list of files.
 \item \verb|:helpgrep pattern [@LANG]| search in the help.
 \item \verb|:helptags /path/to/documentation| for new Vim plugin.
 \end{itemize}

 \section{Adding marks}
 \subsection{Sign, Visible Mark}
 Defining a sign (visible mark): \verb|:sign define {name} {arguments}|, visible
 markers arguments are the signs you want to have, can be linehl, text, texthl 
 or icon. Example could be:
 \begin{verbatim}
 :sign define information text="> linehl=Warning texthl=Error icon=/path
 \end{verbatim}
 Now we have defined sign and have added it to our vimrc file, and we are ready
 to place the sign somewhere:a
 \begin{verbatim}
:exe ":sign place 123 line=".line(".")."name=information file=".expand("%:p")
\end{verbatim}
Replace the number 123 with any number you will use as ID for this sign.

What it does is add the sign named information under the ID 123 to the current
line (line(.)) in the current open file (expand("\%:p")).

Remove the sign: \verb|sign unplace ID|

Remove it only from current file:\verb|sign unplace ID file=name|

See \verb|:help sign| for detail

\subsection{Marks, Hidden Marks}
Simply press the m key in the normal mode followed by one character, for example
\verb|ma| will mark the current line with a. If you
later want to jump to this line, then you simple press \verb|'a|, to jump to
this place instead, simply press \verb|`a|

Note that 0=9 are marks set from \verb|.viminfo| and normally only used by Vim
itself.

You can always get a complete list of your marks by: \verb|:marks|

Delete marks: \verb|:delmarks markid markid...markid|, delete all with
\verb|:delmarks!|
\chapter{Production Boosters}
\section{Template}
Create a directory in our VIMHOME called templates/ and place a file with the
template code (HTML code for example) and save it as \verb|html.tpl|, to get it
loaded to all-new HTML files:
\verb|:autocmd BufNewFile *.html Or $VIMHOME/templates/html.tpl|

Looks for a template that matches the extension of the file:
\begin{verbatim}
:autocmd BufNewFile * silent! Or $VIMHOME/templates/%:e.tpl|
\end{verbatim}
We can add place holder in the templates with \verb|< +KEYWORD+ >|

\section{Snippets with snipMate}
If you want to create a snippet for the for loop:
\begin{verbatim}
snippet for
for(${1:i} = 0; $1 <${2:count}; $1${3:++}) {
    ${4:/*code*/}
}
\end{verbatim}
This text is added to a file placed in \verb|$VIMHOME/snippets| called
\verb|FILETYPE.snippet|

The first line say that a new snippet is started and that it should be executed
when the for word is written and the Tab key is pressed.

The next line is where the real magic goes. It actually starts by writing
writing a snippet of code look like this:
\begin{verbatim}
for(int i=0; i < count; i++) {
    /*code*/
}
\end{verbatim}

However, after the snippet is inserted, it places the cursor in the first i and
goes to into the insert mode. That way you can goes into the insert mode. After
i, you press the Tab key and the cursor will jump to count and again to into the
insert mode. Press Tab again moves to ++ and lastly to the /*code*/ and you can
modify the code.

The snipMate system works by looking for \verb|{NUMBER:INITIAL_VALUE}|. The
NUMBER indicates the sequence in which you will jump to it. If you need the same
value in several places, simply use \verb|$NUMBER|

\section{Using tag list}
\verb|ctags *.c *.h| to generate and \verb|:set tags=/path/to/tags|.

Now Vim knows about the tags. \verb|Ctrl-]| If there is only one match, you are
moved directly to where the function is defined. Or a list of matches if
multiple matches. \verb|Ctrl-t| to where you originally came from

\verb|:tags| to see the stack, and \verb|Ctrl-]| and \verb|Ctrl-t| to move up
and down in the stack. Select by \verb|:tselect|(list of matching) and
\verb|:ptselect|(list of preview window).

\verb|:tnext| and \verb|:tprev| to move to next and previous tag.
\section{Auto-Complete}
\subsection{Autocompletion with known words}
\verb|Ctrl-n| will look up forward of the file. \verb|Ctrl-p| look up backward.

To add dictionary:
\begin{verbatim}
:set dictionary+=/path/to/dictionary/file/with/words
\end{verbatim}
\verb|Ctrl-x| to get into a completion mode and by pressing \verb|Ctrl+k| to do
a lookup for keyword.
\section{Omnicompletion}
It gave the user the possibility to define exactly how the functionality of the
completion should work:
\begin{verbatim}
:set omnifunc=MyCompleteFunction|
\end{verbatim}
Typing in some letters and then going to completion mode by \verb|Ctrl-x|
followed by \verb|Ctrl-o|.
\section{Using macro recording}
\begin{itemize}
    \item \verb|qa| Record from now in into register a;
    \item \verb|q| end the recording.
    \item \verb|@a| executes the recording in register a.
    \item \verb|@@| repeats the last executed command.
\end{itemize}

\section{Using sessions}
Stored information can be split into three categories:
\begin{description}
\item [View] A view can be saved and restored such that a window will have the
same look and setup every time you use the view.
\item [Sessions] Collections of views and information about how they are
interoperating.
\item [All the rest] All the global settings that do not directly apply to any
window in View.
\end{description}
\subsection{Simple session usage}
Save the currently running session
\begin{verbatim}
:mksession FILE
\end{verbatim}
Or save the current view:
\begin{verbatim}
:mkview FILE
\end{verbatim}
Vim uses a file called Session.vim that puts in the current work directory by
default.

Set the path of session files:\verb|:set viewdir=$HOME/.vim/views|

Load a session:\verb|vim -S Session.vim| or with command:
\verb|source Session.vim|, and with view: \verb|:loadview View.vim|




\chapter{spf13-vim and plugins}
 \section{Plug-ins}
 \subsection{NERDTree}
 \begin{itemize}
        \item \verb|<C-E>| to toggle NERDTree
        \item \verb|<leader>e| to load NERDTreeFind which opens NERDTree where 
             the current file is loaded.
 \end{itemize}
 \subsection{ctrlp}
 Ctrlp replaces the Command-T plugin with a 100\% viml plugin. Use \verb|<c-p>| 
 to invoke
 \subsection{Surround}
 This plugin is a tool for dealing with pairs of "surroundings" Examples of 
 surroundings inclue parentheses, quotes, and HTML tags.o
 \subsection{Tagbar}
 spf13-vim includes the Tagbar plugin. This plugin requires exuberant-ctags 
 and will automatically 

 Quickstart: \verb|<C-]>| or \verb|<leader>tt|
 \subsection{Fugitive}
 Adds pervasive git support ot git directories in vim. For more information, 
 \verb|:help fugitive|
\end{document}
