\documentclass[11pt]{book}
\usepackage{parskip}
\usepackage{spverbatim}
\usepackage[top=1.8cm, bottom=2.5cm, left=1.5cm, right=1.5cm]{geometry}
\usepackage{fontspec}
\setmainfont{DejaVu Serif}
\begin{document}
\chapter{Personalizing Vim}
\begin{description}
\item[Changing the font] \verb|set guifont=Courier\ 14| Courier can be exchanged
with the name of any font 14 with any font size.
\item[Changing Color Scheme] \verb|colorscheme mycolor|
\item[personal Highlighting] \verb|match Group /pattern/| Two arguments, the
first one is the name of the color group, the second is the actual pattern you
want to match. use \verb|match NONE| to cancel match, also we have \verb|2match|
\verb|3match| to have second and third match.
\item[Defining color group]
 \verb|hightlight MyGroup ctermbg=red guibg=red gctermfg=yellow guifg=yellow term=bold|
\item[Set how the status line should look] \verb|set statusline format|, format
is a string such as printf that describes how the status line should look. Use
\verb|help stausline| to find the detail. Example
\begin{spverbatim}
set statusline=%F%m%r%h%w\ [FORMAT=%{&ff}]\ [TYPE=%Y]\ [ASCII=\%03.3b]\ [HEX=\%02.2B]\ [POS=%04l,%04v]\ [%p%%]\ [LEN=%L]
\end{spverbatim}
 \item[Make sure status line always shown] \verb|set laststatus=2|
\item[Remove menus in Gvim] \verb|set guioptions-=m|, \verb|set guioptions-=T|, 
see \verb|help 'guioptions'|
\item[Modifying tabs]\verb|set tabline=tabline-layout| and \verb|set guitablabel|
 for gui. It's very similar to one used for the status line.
\item[Remove tab] \verb|set showtabline=0|
\item[Set tab tip] \verb|guitabtooltip|, the value you want to show when the
mouse cursor hovers the tab.
\item[Cursor tracking] \verb|set cursorline|, we can add style
 \verb|highlight CursorLine guibg=lightblue ctermbg=lightgray|, For column:
\verb|set cursorcolumn|
\item[Adding Line number] \verb|set number|, change the default number of
columns used for the line numbers:\verb|set numberwidth=XXX|, XXX is number
\item[Spell Checking]\verb|set spell| and \verb|set spelllang=en,de,da| to set
language. Pressing Z + = to have a list of good guesses.
 \verb|set spellsuggest=X| to set number of alternative ways of spelling
\item[ballons] In Vim, tool tips for the editing area are called
\textbf{ballons} and they are only shown when the cursor is hovering over one of
the characters. The commands you will need to know in order to use the ballon
are
    \begin{itemize}
        \item Turn on \verb|set ballooneval|
        \item Howlong it should wait before showing \verb|set balloondelay=400|
        \item Sets the string that vim will show in the ballon:
            \verb|set balloonexpr="textstring"| can be a static string or return
            from some function
    \end{itemize}
\item[Abbreviations] In vim, abbreviations are created withn:
    \begin{itemize}
        \item \verb|:abbreviate|: abbreviations for all modes
        \item \verb|:labbrev|: abbreviations for insert mode
        \item \verb|:cabbrev|: abbreviations for the command mode only.
    \end{itemize}
    All of the commands take two arguments - the abbreviation and the full text
    should expand to.
    Simply place a file called abbrevation.vim and put in your VIMHOME, and
    simple call \verb|source: $VIM/abbreviation.vim| in your vimrc
\item[Modifying key bindings]
    \begin{itemize}
    \item \verb|:map| for all mode
    \item \verb|:imap| for insert mode only
    \item \verb|:cmap| for command-line mode only
    \item \verb|:nmap| for the Normal mode only
    \item \verb|:vmap| for visual mode only
    \end{itemize}
    e.g Ctrl+S to save:\verb|:map <C-s> :w<cr>| the \verb|<cr>| to execute the
    command. Check \verb|help key-mapping|
\end{description}
 \chapter{Better Navigation}


\end{document}
