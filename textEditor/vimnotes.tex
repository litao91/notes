\documentclass[11pt]{book}
\usepackage{parskip}
\usepackage{spverbatim}
\usepackage[top=1.8cm, bottom=2.5cm, left=1.5cm, right=1.5cm]{geometry}
\usepackage{fontspec}
\setmainfont{DejaVu Serif}
\begin{document}
\chapter{Personalizing Vim}
\begin{description}
\item[Changing the font] \verb|set guifont=Courier\ 14| Courier can be exchanged
with the name of any font 14 with any font size.
\item[Changing Color Scheme] \verb|colorscheme mycolor|
\item[personal Highlighting] \verb|match Group /pattern/| Two arguments, the
first one is the name of the color group, the second is the actual pattern you
want to match. use \verb|match NONE| to cancel match, also we have \verb|2match|
\verb|3match| to have second and third match.
\item[Defining color group]
 \verb|hightlight MyGroup ctermbg=red guibg=red gctermfg=yellow guifg=yellow term=bold|
\item[Set how the status line should look] \verb|set statusline format|, format
is a string such as printf that describes how the status line should look. Use
\verb|help stausline| to find the detail. Example
\begin{spverbatim}
set statusline=%F%m%r%h%w\ [FORMAT=%{&ff}]\ [TYPE=%Y]\ [ASCII=\%03.3b]\ [HEX=\%02.2B]\ [POS=%04l,%04v]\ [%p%%]\ [LEN=%L]
\end{spverbatim}
 \item[Make sure status line always shown] \verb|set laststatus=2|
\item[Remove menus in Gvim] \verb|set guioptions-=m|, \verb|set guioptions-=T|, 
see \verb|help 'guioptions'|
\item[Modifying tabs]\verb|set tabline=tabline-layout| and \verb|set guitablabel|
 for gui. It's very similar to one used for the status line.
\item[Remove tab] \verb|set showtabline=0|
\item[Set tab tip] \verb|guitabtooltip|, the value you want to show when the
mouse cursor hovers the tab.
\item[Cursor tracking] \verb|set cursorline|, we can add style
 \verb|highlight CursorLine guibg=lightblue ctermbg=lightgray|, For column:
\verb|set cursorcolumn|
\item[Adding Line number] \verb|set number|, change the default number of
columns used for the line numbers:\verb|set numberwidth=XXX|, XXX is number
\item[Spell Checking]\verb|set spell| and \verb|set spelllang=en,de,da| to set
language. Pressing Z + = to have a list of good guesses.
 \verb|set spellsuggest=X| to set number of alternative ways of spelling
\item[ballons] In Vim, tool tips for the editing area are called
\textbf{ballons} and they are only shown when the cursor is hovering over one of
the characters. The commands you will need to know in order to use the ballon
are
    \begin{itemize}
        \item Turn on \verb|set ballooneval|
        \item Howlong it should wait before showing \verb|set balloondelay=400|
        \item Sets the string that vim will show in the ballon:
            \verb|set balloonexpr="textstring"| can be a static string or return
            from some function
    \end{itemize}
\item[Abbreviations] In vim, abbreviations are created withn:
    \begin{itemize}
        \item \verb|:abbreviate|: abbreviations for all modes
        \item \verb|:labbrev|: abbreviations for insert mode
        \item \verb|:cabbrev|: abbreviations for the command mode only.
    \end{itemize}
    All of the commands take two arguments - the abbreviation and the full text
    should expand to.
    Simply place a file called abbrevation.vim and put in your VIMHOME, and
    simple call \verb|source: $VIM/abbreviation.vim| in your vimrc
\item[Modifying key bindings]
    \begin{itemize}
    \item \verb|:map| for all mode
    \item \verb|:imap| for insert mode only
    \item \verb|:cmap| for command-line mode only
    \item \verb|:nmap| for the Normal mode only
    \item \verb|:vmap| for visual mode only
    \end{itemize}
    e.g Ctrl+S to save:\verb|:map <C-s> :w<cr>| the \verb|<cr>| to execute the
    command. Check \verb|help key-mapping|
\end{description}

 \chapter{Better Navigation}
 Context aware navigation:
 \begin{itemize}
 \item \verb|{| Go to gthe beginning of the paragraph, or in the empthy line
  just above it, \verb|}| end of the paragraph
 \item \verb|g,| go the place where you make change previously, using this
 comman dseveral times in a row will loop you through locations of previous
 changes in the file.\verb|g;| will let you move forward.
 \item \verb|(:| Move to the beginning of the sentence and \verb|):| end of the
 sentence.
 \item \verb|w| Move to the beginning of the next word.
 \item \verb|b| Move to the beginning of the previous word.
 \item \verb|e| Move to the end of the world.
 \item \verb|%| Go to the other side of the block.
 \item \verb|[[| and \verb|][| Move backwards/ forward to the next section
 beginning(for example, start of a function);
 \item \verb|[]| and \verb|]]| Move backwards/forward to the next section end.
 \item \verb|[{|Move to the beginning of the block, \verb|]}| move to the end of
 the blcok
 \item \verb|[/| Move to the beginning of the comment block, and \verb|]\| for
 the end.
 \item \verb|gd| go to declaraction , start search from current line. The 
 \verb|gD| start from line one. 
 \item \verb|Ctrl+]| to follow the link
 \end{itemize}

 Fater navigation in multiple buffers:
 For every file you open, you have a Vim buffer.
 \begin{itemize}
 \item \verb|:buffers| Bring up the list of buffers and find the right buffer in
 the list. Note that this is not interactive.
 \item With the number N of the buffer where the file is place, you can switch
 to that with: \verb|:buffer N|.
 \item Go cycle through: \verb|bnext| \verb|bprevious|
 \item Open referenced files faster: \verb|gf|
 \end{itemize}
 \chapter{Search and you will find}
 Search In current file:
 \begin{itemize}
 \item \verb|?someWord| the command searches backwards in the file for the first
 word after the question mark.
 \item \verb|/someWord| Search forward.
 \item \verb|g#| and \verb|g*| Now vim not just jumps to the next occurrence of
 the word, but also to anyoccurrences where the world is part of another word.
 \end{itemize}
 Search in multiple files:
 \begin{itemize}
 \item \verb|:vimgrep /pattern/[j][g] file fiel2 ... fileN|, this command takes
 two arguments. The first is the pattern you want to search for. You can use
 Vim's regular expressions in the pattern or you can just write a word. You can
 add either the two flags j and g. The flags help you select how much to get
 your result, and how it should presented to you. If the g flag is added, then
 the result will include a line for each match of the pattern. If the j flag is
 added to the end of your pattern, then you will not be presented with the
 result. I will just be updated into your quickfix list for later retrieval. The
 second argument to vimgrep is the list of files.
 \item \verb|:helpgrep pattern [@LANG]| search in the help.
 \item \verb|:helptags /path/to/documentation| for new Vim plugin.
 \end{itemize}
 X marks the spot(Adding visible and hidden markers).
 \begin{itemize}
 \item \verb|:sign define name arguments|
 \end{itemize}
 \chpater{spf13-vim and plugins}
 \section{Plug-ins}
 \subsection{NERDTree}
 \begin{itemize}
        \item \verb|<C-E>| to toggle NERDTree
        \item \verb|<leader>e| to load NERDTreeFind which opens NERDTree where 
             the current file is loaded.
 \end{itemize}
 \subsection{ctrlp}
 Ctrlp replaces the Command-T plugin with a 100\% viml plugin. Use \verb|<c-p>| 
 to invoke
 \subsection{Surround}
 This plugin is a tool for dealing with pairs of "surroundings" Examples of 
 surroundings inclue parentheses, quotes, and HTML tags.o
 \subsection{Tagbar}
 spf13-vim includes the Tagbar plugin. This plugin requires exuberant-ctags 
 and will automatically 

 Quickstart: \verb|<C-]>| or \verb|<leader>tt|
 \subsection{Fugitive}
 Adds pervasive git support ot git directories in vim. For more information, 
 \verb|:help fugitive|
 
         
\end{document}
