\documentclass[a4paper, 11pt]{book}
\usepackage{parskip}
\usepackage{fontspec}
\usepackage[top=1.8cm, bottom=2.5cm, left=1.5cm, right=1.5cm]{geometry}
\setmainfont{DejaVu Serif}
\title{Backbone JS Application}

\begin{document}
\chapter{Introduction}
\section{Model View Controller}
At the heart of MVC, and the idea is Separated Presentation, it's to make a
clearn division between \emph{domain objects that model our perception of the
real world} and \emph{presentation objects taht are the GUI elements we see on
the screen}

The Domain ojbects should be completely \textbf{self contained} and \textbf{work
without reference to the presentation}.

In MVC, the domain element is referred to as the \emph{model}. Model objects are
completely ignorant of the UI. We'll take teh model as a reaing, with fields for
all the interesting data upon it. 

In MVC, we can assume a domain model of regular objects, rather than the recourd
set. MVC assumes we are manipulating regular Smalltalk objects.

The presentation part of MVC is made of \emph{view} and \emph{controller}

The controller's job is to take the user's input and figure out what to do with
it. The first part of reacting to the user's input is the \emph{various
controllers collaborating to see who got edited}.

Smalltalk figured out that you wanted generic UI components that could be
reused. In this case the component would be \textbf{view-controller pair}.

Both were generic class, so needed to be \textbf{plugged into the application
specific behavior}. There would be an assessment view that would 
\textbf{represent the whole screen and define the layout}.

MVC has no event handlers on the accessment controller for the lower level
components.

A text field example: The configuration of the text field comes from giving it a
link to its model, the reading, and telling it was method to invoke when the
text changes. The text field contorller then makes a reflective invocation of
that method on the reading to make change.

So there is no overall object observing low level widgets, instead \textbf{low
level widgets observe the model}, which itself handles many of the decision that
would be made by the form. For example, when it comes to figuring out the
variance, the reading object itself is the natural place to do that.

Observers do occur in MVC. All the views and controllers observe the model. When
the model changes, the views react. In this case the actual text view is
notified that reading object has changed, and invokes the method to set the
property. 

You'll notice that the text field controller \emph{didn't set the value in the
view itself}, it updated the model and then just let the observer mechanism take
care of the updates. 

MVC does synchronization by making updates on the model and then relying of the
observer relationship to update the views that are oberving the model.

One of the consequences of Oberserver Synchronization is that the controller is
very ignorant of what other widgets need to chagne when the uer manipulates a
particular widget.

This useful ignorance becomes particularly handy if there are multiple screens
open viewing the same model objects. 

The classic MVC example was a spreadsheet liek screen of taht with a couple of
differnt graphs of that data in separate windows. The spreadsheet window didn't
need to be aware of what other windows were open, it just changed the model and
Observer Synchronization took cake of the rest. 

Problem of observer: you can't tell what is happening by reading the code.


\section{MVP}
Both MVC and MVP target the separation of concerns across multiple componetns,
there are some fundamental differences betweens. We will focus on the version of
MVP most suitable fro web-based architecture.
\subsection{Models, View and Presenters}
The P in MVP stands for presenter. It's a component which contains the
\emph{user-interface business logic} for the view.

Unlike MVC, invoctions from the view are delegated to the presenter, which are
decoupled from the view and instead talk to it through an interface. 

The most common implementation of MVP:
\begin{itemize}
\item Uses a Passive View, containing little to no logic.
\item MVP models are almost identical to MVC model models and handle application
data. 
\item The presenter acts as a mediator which talks to both the view and model, 
however both of these are isolated from each other.
\end{itemize}
They effectively bind models to views, a responsibility held by Controllers in
MVC. Presenters are at the heart of the MVP patterns,
ncorporate the presentation logic behind views.

Sociated by a view, presenters perform any work to do with user requests and
pass data  back to them. They retrieve data, manipulate it and determine how the
data should be displayed in the view.

In some implementations, the presenter also interacts with a service layer to
persist data(models).

Models may trigger events but it's the presenter's role to subscribe to them so
that it can update the view. 

In this passive architecture, we have no concept of direct data binding. Views
expose setters which presenters can use to set data.

\section{MVP, MVC and backbone.js}
\begin{itemize}
\item The presenter in MVP better describes the Backbone.View (The layer between
View templates and the data bound to it) than a controller does.
\item The model fits Backbone.Model
\item The views best represent templates (e.g Handle/Mustache markup templates) 
\end{itemize}
The V in MVC and the P in MVP can both be accomplished by Backbone.View because
they are able to achieve two purpose:
\begin{itemize}
\item rendering atomic components
\item Assembling those components rendered by other views.
\end{itemize}

\section{Summary}
Facts about Backbone.js:
\begin{itemize}
\item Core components: Model, View, Collection, Router. Enforces it's own flavor
of MV*
\item Event-driven communication between views and models. It's relatively
straight-forward to add event listeners to any attribute in a model, giving
developers finegrained control over what changes in the view.
\end{itemize}
\chapter{The Basics}
\section{Models}
Backbone models contain \textbf{interactive data} for an application as well as
the \textbf{logic} around this data. 

For example, we can use a model to represent the concept of a photo object
including its attributes like tags, titles and a location:

\begin{verbatim}
var Photo= Backbone.Model.extend({
  defaults: {
    src:'placeholder.jpg',
    title:'an image placeholder',
    coordinates:[0,0]
  },
  initialize: function() {
    this.on("change:src", function(){
      var src = this.get("src");
      console.log('Image source updated to ' + src);
    });
  },
  changeSrc: function(source) {
    this.set({src:source});
  }
});
var somePhoto = new Photo({src:'test.jpg", title:"testing"});
somePhoto.changeSrc("magic.jpg"); //which will triggers "change:src" and logs an
update message to the console
\end{verbatim}
\subsection{Initialization}
The initialize() method is called when a new instance of a model is created
when a new instance of a model is created. 
\subsection{Getters and Setters}
\subsubsection{Model.get()}
\verb|Model.get()| provides easy access to a model's attributes.

Attributes which are passed through to the model on instantiation are instantly
avaliable for retrieval:
\begin{verbatim}
var myPhoto = new Photo({ title: "My awesome photo", 
                      src:"boston.jpg", 
                      location: "Boston", 
                      tags:['the big game', 'vacation']}),

title = myPhoto.get("title"), //My awesome photo
location = myPhoto.get("location"), //Boston
tags = myPhoto.get("tags"), // ['the big game','vacation']
photoSrc = myPhoto.get("src"); //boston.jpg
\end{verbatim}
Alternatively, if you wish to directly access all of the attributes in a model's
instance, you can:
\begin{verbatim}
var myAttributes = myPhoto.attributes;
\end{verbatim}

It is best pratice to use \verb|Model.set()| or direct instantiation to set the
values of a model's attributes.

Accessing \verb|Model.atributes| directly is generally discourage. Instead,
should you need to read or clone data, \verb|Model.toJSON()| is recommended for
this purpose.

\subsection{Model.set()}
\verb|Model.set()| allows us to pass attributes into an instance of our model.
Attributes can be set during initialization or at any time after wards. 

Backbone uses \verb|Model.set()| to know when to broadcast that a model's data
has changed.
\subsection{Default Values}
There are times when you want your model to have a set of default values, this
can be set using a property called \verb|defaults| in your model.
\subsection{Listening for changes to your model}
Any and all of the attributes in a Backbone model can have listeners bound to
them which detect when their values change. Listeners can be added to teh
\verb|initializa()| function:
\begin{verbatim}
this.on('change', function() {
  concole.log('values for this model have changed');
}
\end{verbatim}
For a specific attribute:
\begin{verbatim}
initialize: function(){
    console.log('this model has been initialized');
    this.on("change:title", function(){
        var title = this.get("title");
        console.log("My title has been changed to.. " + title);
    });
},
\end{verbatim}
\subsection{Validation}
Through \verb|Model.validate()|, checking the attribute values for a model prior to them being set.
\begin{verbatim}
var Photo = Backbone.Model.extend({
    validate: function(attribs){
        if(attribs.src === undefined){
            return "Remember to set a source for your image!";
        }
    },

    initialize: function(){
        console.log('this model has been initialized');
        this.on("error", function(model, error){
            console.log(error);
        });
    }
});

var myPhoto = new Photo();
myPhoto.set({ title: "On the beach" });
//logs Remember to set a source for your image!
\end{verbatim}
\section{Creating new Views}
To create a new View, simply extend \verb|Backbone.View|. 
\begin{verbatim}
var PhotoSearch = Backbone.View.extend({
  el: $("#results'),
  render: function(event) {
    var compiled_template =
         _.template($("#results-template").html());
         this.$el.html(
            compiled_template(this.model.toJSON()));
         return this; //recommended as this enables calls to be chained
  }, 
  events: {
    "submit #searchForm": "search",
    "click .reset": "reset",
    "click. advanced": "switchContext"
  },
  search: function(event) {
    //executed when a form '#searchForm' has been submitted.
  },
  reset: function(event) {
    //executed when an element with class "reset" has been clicked
  },
  swithConext: function(event) {
    //executed when an element with class "advanced" has been clicked.
  }
});
\end{verbatim}
\subsection{What is el}
\verb|el| is basically a reference to a DOM element and all views must
 have one. It allows for all of the contents of a view to be inserted 
 into the DOM at once, which makes for faster rendering as browser 
 performs the minimum required reflows and repaints.

 There are two ways to attach a DOM element to a view:
 \begin{itemize}
 \item The element already exists in the page
 \item A new element is created for the view and added maually by the developer.
 \end{itemize}
If the element already exists in the page, you can set \verb|el| as either a CSS
selector that matches the element or a simple reference to the DOM element or a
simple referene to the DOM element.
\begin{verbatim}
el: '#footer',
//OR
el:document.getElementById('footer')
\end{verbatim}
If you want to create a view, set any combination of the following view's
properties:\verb|tagName|, \verb|id| and \verb|className|. A new element will be
created for you by the framework and a reference to it will be available at the
\verb|el| property.
\begin{verbatim}
tagName: 'p',//required, but defaults to 'div' if not set
className: 'container', //optional, you can assign multiple classes to this
property like so 'container homepage'
id: 'header', //optional
\end{verbatim}
The above code creates the \verb|DOMElement| below but doesn't append it to the
DOM. 
\begin{verbatim}
<p id="header" class="container"></p>
\end{verbatim}
\subsection{Understanding render()}
\verb|render()| is an optional function that defines the logic for rendering a
template. 

The \verb|_.tmplate| method in Underscore compiles JavaScript templates into
functions which can be evaluated for rendering.

Next, set the html of the \verb|el| DOM element to the output of processing a
JSON version of the model associated with the view through the compiled
template.
\subsection{The events attribute}
The Backbone \verb|events| attribute allows us to \emph{attach event listeners}
to either custom selectors, or directly to el if no selector is provided. 

An event takes the form:
\begin{verbatim}
{"eventName selector":"callbackFunction"}
\end{verbatim}
and a number of event types are supported, including \verb|click|,
\verb|submit|, \verb|mouseover|, \verb|dblclick| and more.
Backbone uses jQuery's \verb|.delegate()| to provide instant support for event
delegation but goes a little further, extending it so that \verb|this| always
refers to the current view object. 

Any string callback supplied to the events attribute must have a corresponding
function with the same name within the scope of your view.
\section{Collections}
Collections are sets of Models and are created by
extending:\verb|Backbone.Collection|.

Normally, when creating a collection you'll also want to pass through a property
specifying the model that your collection will contain, as well as any instance
properties required. Example:
\begin{verbatim}
var PhotoCollection = Backbone.Collection.extend({
  model:Photo
});
\end{verbatim}

\subsection{Getters and Setters}
Use \verb|Collection.get()| which accepts a single id as follows:
\begin{verbatim}
var skiiingEpicness = PhotoCollection.get(2)|
\end{verbatim}

Sometimes you may also want to get a model based on its \emph{client id}. The
client id is a property that \textbf{Backbone automatically assignes models that
have not yet been saved}. You can get a model's client id from its \verb|.cid|
property.
\begin{verbatim}
var mySkiingCrash = PhotoCollection.getByCid(456);
\end{verbatim}
Backbone Collections don't have setters as such, but do support adding new
models via \verb|.add()| and removing models via \verb|.remove()|

\begin{verbatim}
var a = new Backbone.Model({title:'my vacation'}),
    b = new Backbone.Model({Title:'my holiday'});

var photoCollection = new PhotoCollection([a,b]);
photoCollection.remove([a,b]);
\end{verbatim}
\subsection{Listening for events}
We're able to listen for \verb|add| and \verb|remove| events for when new models
are added 
\begin{verbatim}
var PhotoCollection = new Backbone.Collection();
photoCollection.on("add", function(photo) {
  console.log("I liked " + photo.get("title") + 'its this one, right? '
   + photo.get("src"));
});
PhotoCollection.add([
  {title: "My trip to Bali", src: "bali-trip.jpg"},
  {title: "The flight home", src: "long-flight-oofta.jpg"},
  {tilte: "Uploading pix", src: "too-many-pics.jpg"}
  ]);
\end{verbatim}
In addition, we're able to bind a \verb|change| event:
\begin{verbatim}
PhotoCollection.on("change:title", fcuntion() {
  console.log('there have been updates made to this coolections titles');
});
\end{verbatim}
\section{Fetching models from the server}
\verb|Collection.fetch()| retrieves a default set of models from the server in
the form of a JSON array. When this data returns, the current collection's
contents will be replaced with the collection of the array.
\begin{verbatim}
var PhotoCollection = new Backbone.Collection;
PhotoCollection.url='/photos';
PhotoCollection.fetch();
\end{verbatim}
Under the covers, \verb|Backbone.sync| is the function called every time
Backbone tries to read or save models to the server.

It uses jQuery or Zepto's ajax implementations to make the RESTful request.

In the above example, if we wanted to log an event when \verb|synce()| was
called:
\begin{verbatim}
Backbone.sync = function(method, model) {
  console("I've been passed " + method + "with" + JSON.stringify(model));
};
\end{verbatim}

\end{document}
