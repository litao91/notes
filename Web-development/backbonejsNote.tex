\documentclass[a4paper, 11pt]{book}
\usepackage{parskip}
\usepackage[top=1.8cm, bottom=2.5cm, left=1.5cm, right=1.5cm]{geometry}
\title{Backbone JS Application}
\begin{document}
\chapter{Introduction}
\section{Model View Controller}
At the heart of MVC, and the idea is Separated Presentation, it's to make a
clearn division between \emph{domain objects that model our perception of the
real world} and \emph{presentation objects taht are the GUI elements we see on
the screen}

The Domain ojbects should be completely \textbf{self contained} and \textbf{work
without reference to the presentation}.

In MVC, the domain element is referred to as the \emph{model}. Model objects are
completely ignorant of the UI. We'll take teh model as a reaing, with fields for
all the interesting data upon it. 

In MVC, we can assume a domain model of regular objects, rather than the recourd
set. MVC assumes we are manipulating regular Smalltalk objects.

The presentation part of MVC is made of \emph{view} and \emph{controller}

The controller's job is to take the user's input and figure out what to do with
it. The first part of reacting to the user's input is the \emph{various
controllers collaborating to see who got edited}.

Smalltalk figured out that you wanted generic UI components that could be
reused. In this case the component would be \textbf{view-controller pair}.

Both were generic class, so needed to be \textbf{plugged into the application
specific behavior}. There would be an assessment view that would 
\textbf{represent the whole screen and define the layout}.

MVC has no event handlers on the accessment controller for the lower level
components.

A text field example: The configuration of the text field comes from giving it a
link to its model, the reading, and telling it was method to invoke when the
text changes. The text field contorller then makes a reflective invocation of
that method on the reading to make change.

So there is no overall object observing low level widgets, instead \textbf{low
level widgets observe the model}, which itself handles many of the decision that
would be made by the form. For example, when it comes to figuring out the
variance, the reading object itself is the natural place to do that.

Observers do occur in MVC. All the views and controllers observe the model. When
the model changes, the views react. In this case the actual text view is
notified that reading object has changed, and invokes the method to set the
property. 

You'll notice that the text field controller \emph{didn't set the value in the
view itself}, it updated the model and then just let the observer mechanism take
care of the updates. 

MVC does synchronization by making updates on the model and then relying of the
observer relationship to update the views that are oberving the model.

One of the consequences of Oberserver Synchronization is that the controller is
very ignorant of what other widgets need to chagne when the uer manipulates a
particular widget.

This useful ignorance becomes particularly handy if there are multiple screens
open viewing the same model objects. 

The classic MVC example was a spreadsheet liek screen of taht with a couple of
differnt graphs of that data in separate windows. The spreadsheet window didn't
need to be aware of what other windows were open, it just changed the model and
Observer Synchronization took cake of the rest. 

Problem of observer: you can't tell what is happening by reading the code.


\section{MVP}
Both MVC and MVP target the separation of concerns across multiple componetns,
there are some fundamental differences betweens. We will focus on the version of
MVP most suitable fro web-based architecture.
\subsection{Models, View and Presenters}
The P in MVP stands for presenter. It's a component which contains the
\emph{user-interface business logic} for the view.

Unlike MVC, invoctions from the view are delegated to the presenter, which are
decoupled from the view and instead talk to it through an interface. 

The most common implementation of MVP:
\begin{itemize}
\item Uses a Passive View, containing little to no logic.
\item MVP models are almost identical to MVC model models and handle application
data. 
\item The presenter acts as a mediator which talks to both the view and model, 
however both of these are isolated from each other.
\end{itemize}
They effectively bind models to views, a responsibility held by Controllers in
MVC. Presenters are at the heart of the MVP patterns,
ncorporate the presentation logic behind views.

Sociated by a view, presenters perform any work to do with user requests and
pass data  back to them. They retrieve data, manipulate it and determine how the
data should be displayed in the view.

In some implementations, the presenter also interacts with a service layer to
persist data(models).

Models may trigger events but it's the presenter's role to subscribe to them so
that it can update the view. 

In this passive architecture, we have no concept of direct data binding. Views
expose setters which presenters can use to set data.
\end{document}
