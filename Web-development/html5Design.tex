\documentclass[a4paper, 12pt]{book}
\usepackage[top=2.5cm, bottom=1.8cm, left=1.5cm, right=1.5cm]{geometry}
\usepackage{parskip}
\usepackage{fontspec}
\usepackage{graphicx}
\setmainfont{DejaVu Serif}
\begin{document}
\chapter{The markup}
As a very basic start to our markup, this is our html file skeleton:
\begin{verbatim}
<!DOCTYPE html>
<html lang="en">
<head>
<meta charset="utf-8" />
<title>Smashing HTML5!</title>

<link rel="stylesheet" href="css/main.css" type="text/css" />

 <!--[if IE]>
 <script src="http://html5shiv.googlecode.com/svn/trunk/html5.js"></script> <![endif]-->
 <!--[if lte IE 7]> <script src="js/IE8.js" type="text/javascript"></script><![endif]-->
 <!--[if lt IE 7]>

<link rel="stylesheet" type="text/css" media="all" href="css/ie6.css"/><![endif]-->
</head>  <body id="index" class="home">
</body>
</html>

\end{verbatim}
A few hightlight:
\begin{itemize}
\item 3 different \emph{Conditional Comments} for IE. First one includes html5
shiv code directly from Google Code for all version of IE. The css files for IE
solve some bugs of IE.
\item The use of \verb|index| id and a \verb|home| class on \verb|<body>| tag.
This is just a habbit.
\item A simplified version of the charset property for better backwards
compatibility with legacy browsers.
\end{itemize}
\section{The header}
Use the new \verb|<header>| tag, the spec of it as follows:
\section{Header:} The header element represents a group of introductory or
navigational aids:

We will also use the \verb|<nav>| tag:
\section{nav:} The \verb|nav| element represents a section of a page that
links to other pages or to parts within the page: a section with navigation
links. Not all groups of links on a page need to be in nav element -- only
sections that consist of major navigation blocks are appropriate for the nav
element.

After a couple of id's and classes our header ends up like this
\begin{verbatim}
<header id="banner" class="body">
 <h1><a href="#">Smashing HTML5!
<strong>HTML5 in the year <del>2022</del>
<ins>2009</ins></strong></a></h1>

<nav><ul>
<li class="active"><a href="#">home</a></li>
<li><a href="#">portfolio</a></li>
<li><a href="#">blog</a></li>
<li><a href="#">contact</a></li>
</ul></nav>
</header><!-- /#banner -->
\end{verbatim}
\section{Featured block}
This is best marked up as an \verb|<aside>|:
\section{aside:} The aside element represents a section of a page that
consists of content that is tangentially related to the content around the aside
element, and which could be considered separate that content. Such sections are
often represented as sidebars in printed typography

That pretty much sums up the featured block.  Now, inside of this block there's
a lot going on. Firstly, this is an article, so alongside the \verb|<aside>|
tag, we should be using \verb|article>| right away.

We also have two consecutive headings, so we'll be using \verb|<hgroup>|. This
is a wonderful tag used for grouping series of \verb|<h#>| tags. It exist to
\emph{mask an h2 element} from the outline algorithm.

The last element on this block is the logo to the right. We have another
tag \verb|<figure>|. The tag is used to enclose some flow content, optionally
with a caption, that is self-contained and is typically refereneced as a single
unit from the main flow of the document. This tag allows us to use a
\verb|<legned>| tag to add a caption to the elemnt inside.

The feature block will look like:
\begin{verbatim}
<asde id="featured" class="body"><article>
<figure>
<img src="images/temp/sm-logo.gif"
alt="Smashing Magazine" />
</figure>
<hgroup>

<h2>Featured Article</h2>
<h3><a href="#">HTML5 in Smashing Magazine!</a></h3>
</hgroup>
<p>Discover how to use Graceful Degradation and Progressive
Enhancement techniques to achieve an outstanding, cross-browser <a
href="http://dev.w3.org/html5/spec/Overview.html" rel="external">HTML5</a> and
<a href="http://www.w3.org/TR/css3-roadmap/" rel="external">CSS3</a> website
today!</p>

 </article></aside><!-- /#featured -->
\end{verbatim}
\section{The layout's body}
Since the block represents a generic document section and a section is a
thematic group of content, this one is a \verb|<section>| tag.

We'll use the onld \verb|<ol>| tag. Each \verb|<li>| should have an
\verb|<article| tag and within this, we'll have a \verb|<header>| for
\textbf{post title}, a footer for the \textbf{post information} and a
\verb|<div>| for the \textbf{post content}. We'll use a \verb|<div>| since it
provides no semantic value by itself and keeps the markup as clean as possible.
And we'll be using the hAtom 0.1 Microformat.

The code shall look like this:
\begin{verbatim}
<section id="content" class="body">
<ol id="posts-list" class="hfeed">

<li>
<article class="hentry">
<header>
<h2 class="entry-title">
    <a href="#" rel="bookmark" title="Permalink to this POST TITLE">This be the title
</a></h2>
</header>

<footer class="post-info">
 <abbr class="published"
title="2005-10-10T14:07:00-07:00"><!-- YYYYMMDDThh:mm:ss+ZZZZ --> 10th October
2005 </abbr>
<address class="vcard author"> By <a class="url fn"
href="#">Enrique Ramírez</a>  </address>
</footer>
<!-- /.post-info -->

 <div class="entry-content">
<p>Lorem ipsum dolor sit amet, consectetur adipiscing
elit. Quisque venenatis nunc vitae libero iaculis elementum. Nullam et justo
<a href="#">non sapien</a> dapibus blandit nec et leo. Ut ut malesuada tellus.</p>
</div><!-- /.entry-content -->
</article></li>
 <li><article class="hentry"> ...
</article>
</li>
<li><article class="hentry"> ... </article></li>
 </ol><!-- /#posts-list -->
 </section><!-- /#content -->
\end{verbatim}
\section{The Extra Block}
This section could not be considered separate from the main content since it
contains the blogroll links and some social information of the website. Thus a
\verb|section>| element is more appropriate.

The code will be like:
\begin{verbatim}
<section id="extras" class="body">
<div class="blogroll">
 <h2>blogroll</h2>
 <ul>
<li><a href="#" rel="external">HTML5 Doctor</a></li>
<li><a href="#" rel="external">HTML5 Spec (working draft)</a></li>
<li><a href="#" rel="external">Smashing Magazine</a></li>
<li><a href="#" rel="external">W3C</a></li>
<li><a href="#" rel="external">Wordpress</a></li>
<li><a href="#" rel="external">Wikipedia</a></li>
...
</ul>
</div><!--/.blogroll -->
<div class="social">
<h2>social</h2>
<ul>
<li><a href="http://delicious.com/enrique_ramirez" rel="me">delicious</a></li>
<li><a href="http://digg.com/users/enriqueramirez" rel="me">digg</a></li>
<li><a href="http://facebook.com/enrique.ramirez.velez" rel="me">facebook</a></li>
...
</ul>
</div><!--/.social -->
</section><! --/#extras -->
\end{verbatim}
\section{About And Fotter Blocks}
We'll use the new \verb|<footer>| tag to wrap both the about and the copyright
information:
\section{footer:} The footer element represents a footer for its nearest
ancestor sectioning content. A footer typically contains information about its
section such as who wrote it, links to related documents, copyright data, and
the like.

For the about block, we'll use an \verb|<address>| tag, which contains contact
information for it's nearest \verb|<article>| or \verb|<body>| element ancestor.
\begin{verbatim}
<footer id="contentinfo" class="body">
<address id="about" class="vcard body">
<span class="primary">
<strong><a href="#" class="fn url">Smashing Magazine</a></strong>

<span class="role">Amazing Magazine</span>
</span><!--/.primary-->

<img src="images/avatar.gif" alt="Smashing Magazine Logo" class="photo" />
<span class="bio">Smashing Magazine is a website and blog that offers resources and
advice to web developers and web designers. It was founded by Sven Lennartz and
Vitaly Friedman.</span>
</address><!-- /#about --> <p>2005-2009
 <a href="http://smashingmagazine.com">Smashing Magazine</a>.
</p>
</footer><!-- /#contentinfo -->
\end{verbatim}
\chapter{The CSS}
A basic start:
\begin{verbatim}
/* Imports */
@import url("reset.css")
@import url("global-forms.css")
 /***** Global *****/
 /* Body */
body { background: #F5F4EF url('../images/bg.png');
color: #000305
font-size: 87.5% /* Base font size: 14px */
font-family: 'Trebuchet MS', Trebuchet,
 'Lucida Sans Unicode', 'Lucida Grande',
'Lucida Sans', Arial, sans-serif;
line-height: 1.429
margin: 0
padding: 0
text-align: left;
}
/* Headings */
h2 {font-size: 1.571em} /* 22px */
h3 {font-size: 1.429em} /* 20px */
h4 {font-size: 1.286em} /* 18px */
h5 {font-size: 1.143em} /* 16px */
h6 {font-size: 1em}  /* 14px */
 h2, h3, h4, h5, h6 {
font-weight: 400;
line-height: 1.1;
margin-bottom: .8em;
}

/* Anchors */
a {outline: 0;}
a img {border: 0px;
text-decoration: none;}
 a:link, a:visited {
color: #C74350
padding: 0 1px;
text-decoration: underline;
}

a:hover, a:active
{ background-color: #C74350;
color: #fff;
text-decoration: none;
text-shadow: 1px 1px 1px #333;
}

/* sections */
p {margin-bottom: 1.143em;}
* p:last-child {margin-bottom: 0;}

strong, b {font-weight: bold;}
em, i {font-style: italic;}

::-moz-selection {background: #F6CF74;
color: #fff;}

::selection {
background: #F6CF74;
color: #fff;
}
/* Lists */
ul {
list-style: outside disc;
margin: 1em 0 1.5em 1.5em;
}

ol {
list-style: outside decimal;
margin: 1em 0 1.5em 1.5em;
}
dl {margin: 0 0 1.5em 0;}
dt {font-weight: bold;}
dd {margin-left: 1.5em;}
/* Quotes */
blockquote {font-style: italic;}
cite {}  q {}
/* Tables */
table {
margin: .5em auto 1.5em auto;
width: 98%;
}

/* Thead */
thead th {padding: .5em .4em text-align: left;}
thead td {}

/* Tbody */
tbody td {padding: .5em .4em;}
tbody th {}
tbody .alt td {}
tbody .alt th {}

/* Tfoot */
tfoot th {}
tfoot td {}
\end{verbatim}

This is the first step into getting the layout together. We can style most of
the basic elements from here. A few highlights:
\begin{itemize}
\item Two import, the first is Eric Meyer's CSS reset file. Second one is
personalized global forms file.
\item Very basic styling for the default tags.
\end{itemize}

\section{Explaining some properties}
Text shadow: \verb|text-shadow: 1px 5px 2px #333;|, this will give a #333 shadow
on our text that's 1px to the right, 5px down and with 2px blur.

We also have: \verb|* p:last-child {margin-bottom: 0;}|
This line will remove the margin bottom of any \verb|<p>| tag that's the last
child of it's parent. Useful when using boxes to avoid large vertical gaps.

Couple of selectors:
\begin{verbatim}
::-moz-selection {background: #F6CF74; color:#fff;}
::selection {background: #F6CF74; color: #fff;}
\end{verbatim}
\verb|::selection| is a CSS3 selector that lets us style how the text section
looks. It only allows color and background CSS properties, so keeps it simple.

\section{Enabling HTML5 Elements}
Display the unknown tags like a \verb|<div>|:
\begin{verbatim}
header, section, footer,
aside, nav, article, figure {
    display: block;
}
\end{verbatim}

\section{Limiting blocks}
We add the \verb|class="body"| attribute to the major sections of the layout in
the markup. This is because we want the body of the websiite to be for a certain
width (800px). So we'll use the basic block centering technique using margins
for this. Also add a couple of generic classes to this section that might be
used for a post side content:
\begin{verbatim}
/** Layout *****/
.body {center: both; margin: 0 auto; width: 800px;}
img.right figure.right {float: right; margin 0 0 2em 2em;}
img.left, fighre.left {float: right; margin: 0 0 2m 2em;}
\end{verbatim}

\section{Header Style}
Begin with our header. We just want a couple of spacing and a few text style
here and there.
\begin{verbatim}
/* main Nav */
#banner nav {
    background: #000305;
    font-size: 1.143em;
    height: 40px;
    line-height: 30px;
    margin: 0 auto 2em auto;
    padding: 0;
    text-align: center;
    width: 800px;

    boder-radius: 5px;
    -moz-border-radius: 5px;
    -webkit-border-radius: 5px;
}

#banner nav ul {
    list-style: none;
    margin: 0 auto;
    width: 800px;
}

#banner nav li {
    float: left;
    display: inline;
    margin:0
}

#banner nav a: link, #banner nav a:visited {
    color: #fff;
    display: inline-block;
    height:30px;
    padding: 5px 1.5em;
    text-decoration:none;
}

#banner nav a:hover, #banner nav a:active,
#banner nav .active a:link, #banner nav .active
a:visited {
    background: #C74451;
    color: #fff;
    text-shadow: none !important;
}

#banner nav li:first-child a{
    border-top-left-radius: 5px;
    -moz-border-radius-topleft: 5px;
    -webkit-border-top-left-radius: 5px;
}
\end{verbatim}

Another CSS3 property here:\verb|border-radius|. This new property lets us add
rounded borders to our blocks without the need of unnecessary non-semantic tags
that will clutter our code or a million of images and clever
background-positioning.

The \verb|!important| is basically to override the default styles without
complex specificity selectors.


\end{document}
