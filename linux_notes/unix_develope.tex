\documentclass[a4paper,11pt]{book}
\usepackage{parskip}
\usepackage{amsthm}
\usepackage{amsmath}
\usepackage{amsfonts}
\topmargin -1.5cm
\oddsidemargin -0.04cm
\evensidemargin -0.04cm
\textwidth 16.59cm
\textheight 25cm
\newtheorem{theorem}{Theorem}[chapter]
\newtheorem{lemma}[theorem]{Lemma}
\newtheorem{proposition}[theorem]{Proposition}
\newtheorem{corollary}[theorem]{Corollary}
\title{Unix envirnoment development}
\begin{document}
\chapter{Unix System Overview}
\section{Logging in}
\subsection{Login Name}
The system looks up login name in its password file, usually the file
\verb|/etc/passwd|.

The password file is composed of seven colon-separated fields:
\begin{itemize}
\item the login name
\item encrypted password
\item numeric user ID
\item numeric group ID
\item Comment field
\item home directory
\item shell program
\end{itemize}
\subsection{Shells}
A shell is a command-lien interpreter that reads user input and executes
commands.

The user input to a shell is normally from the terminal (an interactive shell)
or sometimes from a file.
\section{Files and Directories}
\subsection{File System}
The UNIX file system is a hierarchical arrangement of directories and files.
Everything starts in the directory called root whose name is the single
character /.

A directory is a file that contains directory entries.

Logically, we can think of each directory entry as containing a filename along
with a structure of information describing the attributes of the file. 

\subsection{Filename}
The names in a directory are called filenames.

Two filenames are automatically created whenever a new directory is createdi .
(called dot) and .. (called dot dot). Dot refers to the current directory, and
dot-dot refer to the parent directory. \textbf{In the root directory, the
dot-dot is the same as the dot}. 

\subsection{pathname}
A sequence of one or more filenames, separated by slashes and optionally
starting with a slash, forms a \textbf{pathname}.

A pathname begins with a slash is called a \textbf{absolute pathname}.
Otherwise, it's called an \textbf{absolute pathname}.



\end{document}
