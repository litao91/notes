\documentclass[11pt, a4paper]{book}
\usepackage{parskip}
\usepackage[top=1.5cm, left=1cm, right=1cm, bottom=2.5cm]{geometry}
\usepackage{fontspec}
\usepackage{graphicx}
\setmainfont{Roboto}
\setmainfont{DejaVu Serif}

\begin{document}
\chapter{Overview}
\section{Ruby is Object-Oriented}
Every value is an object, even simple numeric literals and the values true,
false and nil.

Example:Invoke a method named class on these values, comments begins with
\verb|#| in ruby, the \verb|=>| arrows in the comments indicated the value
returned by the code:
\begin{verbatim}
1.class # => Fixnum: the number 1 is a fix num
true.class # => TrueClass: true is the singleton instance of TrueClass
\end{verbatim}
In Ruby, parentheses are usually optional and they are commonly omitted.
\section{Blocks and Iterators}
\begin{verbatim}
3.times{ print "Ruby!"} # Prints "Ruby!Ruby!Ruby"
1.upto(9){|x| print x}  # Prints "123456789"
\end{verbatim}
Times and upto are methods implemented by integer objects. They are special kind
of method known as iterator, and they behave like loops.

The code within curly braces known as a block is associated with the method
invocation and serves as the body of the loop.

Arrays define an iterator named each:
\begin{verbatim}
a = [3, 2, 1]
a.each do |elt| #each is an iterator. The block has a parameter elt.
    print elt+1
end #this block was delimited with do/end instead of {}
\end{verbatim}

Various other iterators are defined upon each:
\begin{verbatim}
a = [1, 2, 3, 4]
b = a.map{|x| x*x} #square elements, b is [1, 4, 6, 9]
c = a.inject do |sum, x| #compute the sum of elemetns
    sum + x
end
\end{verbatim}
Hashes are fundamental data structure in Ruby, they use square brackets.
\begin{verbatim}
h = {   # A hash that maps numbers to digits
    :one => 1, # The "arrows" show mappings: key => value
    :two => 2
}
h[:one]    # => 1. Access a value by key
h[:three] = 3 # Add a new key/value pair to the hash.
h.each do |key, value| #iterate through the key/value pairs
    print #{value}:#{key};" #Note variables substitued into string
end
\end{verbatim}
Keys can by any objects, buy Symbol objects are most commonly used.

Symbols are \emph{immutable, interned strings}. They can be compared by
identity rather than by textual content.

The ability to \emph{associate a block of code with a method invocation} is
fundamental. 

Double-quoted strings can include arbitrary Ruby expressions delimited by
\verb|#{\ and\ }|. The value of the expression within these delimiters is
converted to string (by calling its \verb|to_s| method, which is supported to
all objects.
\section{Expressions and Operators in Ruby}
Statements in other language are actually expressions in Ruby.
\begin{verbatim}
minimum = if x < y then x else y end
2 ** 1024 # 2 to the power of 1024
"Ruby" + "rocks!" # Stirng concatenation
"Ruby!"*3 #String repetition
"%d %s" % [3, "rubies"] # => "3 rubies": Py style, printf formatting
\end{verbatim}
\section{Methods}
Methods are defined with \verb|def| keyword:
\begin{verbatim}
def square(x)
    x*x
end
\end{verbatim}
When a method is defined outside of a class or module, it is effectively a
global function(Technically becomes a private method of the Object class.

Methods can also be defined on individual objects by prefixing the name of the
method with the object on which it is defined:
\begin{verbatim}
def Math.square()
    x*x
end
\end{verbatim}
\section{Assignment}
Ruby supports parallel assignment:
\begin{verbatim}
x, y = 1, 2 # same as x =1; y=2
a, b = b, a # swap the values
x, y, z  = [1, 2, 3] #Array elements automatically assigned to variables.
\end{verbatim}
Methods in Ruby are allowed to return more than one value, and parallel
assignment is helpful in conjunction with such methods:
\begin{verbatim}
def polar(x, y)
    theta = Math.atan2(x,y) # compute the angle
    r = Math.hypot(x,y) # compute the distance
    [r, theta] # The last expression is the return value
\end
\end{verbatim}

Methods whose names end with = can be invoked by assignment expressions:
\begin{verbatim}
o.x=(1)  #o has a method named x=
o.x = 1  #x= method invoked by assignment expression
\end{verbatim}

End with a question mark: mark predicates that are return a Boolean value,
Exclamation mark at the end of a method name is used to indicate hat caution is
required with the use of the method. Usually the method without exclamation mark
returns a modified copy of the object and the one with the exclamation mark is a
mutator method that alters the object in place. 

Global variables are prefixed with \verb|$|, instance variables are prefixed
with \verb|@|, and class variables are prefixed with \verb|@@|

\section{Regexp and Range}
A Regexp object describes a texture pattern and has methods for determining
whether a given string matches that pattern or not.

A Range represents the values between two endpoints. Regular expressions and
ranges have a literal syntax in ruby
\begin{verbatim}
/[Rr]uby/
/\d{5}/  #Matches 5 onsecutive digits
1..3     # All x where 1 <= x <=3
1...3    # All x where 1<= x < 3
\end{verbatim}
Regexp and Range objects define the normal == operator for testing equality.
They also define the === operator for testing matching and membership.

\section{Classes and Modules}
An object's state is held by its instance variables, which begin with \verb|@|
and whose vales are specific to that particular object.
\begin{verbatim}
class Sequence
    def initialize(from, to, by)
        @from, @to, @by = from, to, by #Note parallel assignment and @ prefix
    end

    def each
        x = @from
        while x<= @to
            yield x           # Pass x to the block associated with the iterator
            x += @by
        end
    end

    def length
        return 0 if @from > @to #Note if used as a statement modifier
        Integer((@to-@from)/@by) + 1
    end
    alias size length # size is now a synonym for length
    #Override array-access operator to give random access to the sequence
    def[](index)
        return nil if index < 0
        v = @from + index*@by
        if v <= @to
            v
        else
            nil
        end
    end
    # Override arithmetic operators to return new Sequence objects
    def *(factor)
        Sequence.new(@from*factor, @to*factor, @by*factor)
    end
end
\end{verbatim}
Here is some code that uses this \verb|Sequence| class:
\begin{verbatim}
s = Sequence.new(1,10,2)
s.each{ |x|print x}
print s[s.size - 1]
t = (s + 1) * 2
\end{verbatim}
If we just want to write the iterator, we can define it in a model of its own:
\begin{verbatim}
module Sequences                   # Begin a new module
  def self.fromtoby(from, to, by)  # A singleton method of the module
    x = from
    while x <= to
      yield x
      x += by
    end
  end
end
\end{verbatim}

Even the built-in core classes are open: any program can add method to them:
\begin{verbatim}
class Range                  # Open an existing class for additions
  def by(step)               # Define an iterator named by
    x = self.begin           # Start at one endpoint of the range
    if exclude_end?          # For ... ranges that exclude the end
      while x < self.end     # Test with the < operator
        yield x
        x += step
      end
    else                     # Otherwise, for .. ranges that include the end
      while x <= self.end    # Test with <= operator
        yield x
        x += step
      end
    end
  end                        # End of method definition
end                          # End of class modification

# Examples
(0..10).by(2) {|x| print x}  # Prints "0246810"
(0...10).by(2) {|x| print x} # Prints "02468"
\end{verbatim}
\section{Ruby Surprise}
\begin{itemize}
    \item Ruby's strings are mutable, \verb|[]=| allows you to alter the
    characters of a string or to insert, \verb|<<| operators allows you to
    append to a string, call the freeze method on a string to prevent any future
    modifications.
    \item nil is false, and any other is true

\end{itemize}
\section{Try Ruby}
\verb|puts| print string with a newline, call \verb|to_s| is not string.
\verb|print| does not append a newline. \verb|p| is a good alternative to
\verb|puts|, it will converts objects to strings with the inspect method.
\verb|irb|, interactive Ruby. \verb|ri| documentation viewer. ri followed by the
name of a Ruby class, module, or method.
\chapter{The structure and execution of Ruby}
\section{Datatype and Objects}
\subsection{Hash Literals}
A hash literal is written as a comman-separated list of key/value pairs,
enclosed within curly braces. Keys and values are separated with a two-character
"arrow": \verb|=>|
\begin{verbatim}
numbers = { "one" => 11, "two" => 2, "three" => 3 }
\end{verbatim}

The \verb|Hash| class compares keys for equality with the \verb|eql?| method. If
you define a new class that override the \verb|eql?| method, you must also
override the \verb|hash| method. 

If you define a class and do not override \verb|eql?|, then instances of that
class are compared for object identity when used as hash keys. \emph{Two
distinct instances of your class are distinct hash keys even if they represent
the same content}. In this case, the default \verb|hash| method is appropriate:
it returns the unique \verb|object_id| of the object.

Note that mutable objects are problematic as hash keys. Changing the content of
an object typically changes its hashcode. If you use an object as a key and then
alter that object, the internal hash table becomes corrupted, and the hash no
longer works correctly.

\subsection{Ranges}
A Range object represents the values between a start value and an end value.
Range literals are written by placing two or three dots between the start and
end value. If two dots are used, then the range is inclusive and the end value
is part of the range. If three dots are used, then the range is exclusive and
the end value is not part of the range.

Test whether a value is included in a range with the \verb|include?| method.

Implicit in the definition of a range is the notion of ordering. If a range is
the values between two endpoints, there obviously must be some way to compare
values to those endpoints. In ruby, this is done with the comparison operator
\verb|<=>|, which \emph{compares its two operands and evaluates to -1, 0, or 1},
depending on their relative order(or equality). Classes such as numbers and
strings that have an ordering define the \verb|<=>| operator. A value can only
be used as a range endpoint i it responds to this operator..

Technically, any value that is compatible with the \verb|<=>| operators of the
range endpoints can be considered a member of the range.

The primary purpose for ranges is comparison: to be able to determine whether a
value is in or out of the range.

An important secondary purpose is iteration: if the class of the endpoints of
the range defines a \verb|succ| method, then there is a discrete set of range
members, and they can be iterate with \verb|each|, \verb|step|, and
\verb|Enumerate|(\verb|to_a| to array) method.

It is necessary to explain that range membership can be defined in two different
ways.

All the range endpoint values must implement the \verb|<=>| operator, so this definition
of membership works for any Range object and does not require the endpoints to
implement the \verb|succ| method. We'll call this \textbf{continuous membership
test.}

The second definition of membership is \textbf{discrete membership}, which does
depend on \verb|succ|. It treats a \verb|Range| \verb|begin...end| as a set that
include \verb|begin|, \verb|begin.succ|, \verb|begin.succ.succ|, and so on. By
this definition, range membership is set membership, and the value x is included
in the range only if it is a value returned by one of the \verb|succ|.

In Ruby 1.9, \verb|cover?| always uses the continuous membership test.
\verb|include?| and  \verb|member?| are  synonyms and they are use the
continuous membership test for numbers. If the endpoint are not numeric, however they
instead use the discrete membership test.

\subsection{Symbols}
A symbol literal is written by prefixing an identifier or string with a column:
\verb|:symbol|, symbol also have a \verb|%s| literal syntax that allows
arbitrary delimiters

Symbols are \textbf{often used to refer to method names in reflective code}. For
example: \verb|o.respond_to?:each|

Another example, tests whether a give object responds to a specified method, and
if so, invoke it:
\begin{verbatim}
name = :size
if o.respond_to? name
    o.send(name);
}
\end{verbatim}
You can convert a string to a symbol by \verb|intern| or \verb|to_sym|, symbol
to string with \verb|to_s|

Two strings with the same content will \textbf{always convert to exactly the
same Symbol object}.

\subsection{True, False and Nil}
When Ruby requires a Boolean value, nil behaves like false, and any value other
than nil or false behaves like true.

\subsection{Objects}
Ruby is a very pure object-oriented language: all values are objects, and there
is no distinction between primitive types and object type.

All objects inherit from a class named Object and share the methods defined by
that class.

\subsubsection{Object Reference}
When we work with object in Ruby, we are really working with object references.
It is not the object itself we manipulate but a reference to it. When we assign
a value to a variable, we are not copying an object into that variable; we are
merely storing a reference to that variable.

When you pass an object to a method in Ruby, it is an object reference that is
passed to the method. It is not the object itself, and \emph{it is not a
reference to the reference to the object}. So it's passed by value but rather
than by reference, but that values passed are object references. So method can
use those references to modify the underlying object.
\subsubsection{Immediate values}
Fixnum and Symbol objects are actually immediate values rather than references
(they are immutable)

\subsubsection{Object Identity}
Every object has an object identifier, by \verb|object_id| method.\verb|__id__|
is a synonym for \verb|object_id|

\chapter{Statements and Control Structures}
\section{Conditions}

\subsection{if}
\begin{verbatim}
if <expression>
    <code>
end
\end{verbatim}
or
\begin{verbatim}
if <expression> then <code> end
\end{verbatim}
\begin{itemize}
    \item Parentheses are not required around the conditional expression. The
    new line, semicolon, or then keyword serves to delimit the expression
    instead.
    \item The end keyword is required
\end{itemize}
\subsection{else and elsif}
\begin{verbatim}
if <expression>
    <code>
else
    <code>
end
\end{verbatim}

and:
\begin{verbatim}
if <expression1>
    <code1>
elsif <expression2>
    <code2>
...
elsif <expressionN>
    <codeN>
else
    code
end
\end{verbatim}
\subsection{Return value}
The control structures that are commonly called statements. The return value of
an if statement is the value of the last expression in the code that was
executed, or nil if no block of code was executed or nil if no block of code was
executed.
\subsection{If As a Modifier}
Instead of writing \verb|if <expression> then <code> end| we can simply write
\verb|<code> if <expression>|, when in this form, if is known as a statement.
Even though the condition is written last, it is evaluated first.

\section{Loop}
\subsection{while and until}
\begin{verbatim}
x = 10
while x>= 0 do
    puts x
    x = x - 1
end

x = 0
until x > 10 do
    puts x
    x = x + 1
end
\end{verbatim}
The while loop evaluates its condition. If the value is anything other than
false or nil, it executes its body, and then loops to evaluate its condition
again.

The until loop is the reverse. The condition is tested and the body is executed
if the condition evaluates to false or nil.
\subsection{While and until as modifiers}
\begin{verbatim}
x = 0
puts x = x + 1 while x < 10

x = 0
whiel x < 0 do puts x = x + 1 end


a = [1, 2, 3]
puts a.pop until a.empty?
\end{verbatim}

Not that when while and until are used as modifiers, they must appear on the
same line as the loop body that they modify. The loop condition is tested first,
even though it is written after the loop body.

There is a special-case exception to this rule. When the expression being
evaluated is a compound expression delimited by begin and end keywords, then the
body is executed first before the condition is tested:
\begin{verbatim}
x = 10
begin
    puts x
    x = x - 1
end until x == 0
\end{verbatim}
\subsection{The for/in loop}
The for loop iterates through the elements of an enumerable object. On each
iteration, it assigns an element to a specified loop variable and then executes
the body of the loop:
\begin{verbatim}
for <var> in <collection> do
    <body>
end
\end{verbatim}
\verb|var| is a variable or comma-separated list of variables.
\verb|collection| is any object that has an \verb|each| iterate method. The
for/in loop calls the each method of the specified object. As that yields
values, the for loop assigns each value to the specified variable and then
executes the code in the body.
\begin{verbatim}
hash = {:a=>1, :b=>2, :c =>3}
for key, value in hash
    puts "#{key} => #{value}"
end
\end{verbatim}
The loop variable or variables of a for loop are \emph{not local to the loop
exists}. Similarly, new variables defined within the body of the loop continue
to exist after the loop exits.

For loop are much like iterators. The above is equivalent:
\begin{verbatim}
hash = {:a=>1, :b=>2, :c=>3}
hash.each do |key, value|
    puts "#{key} => #{value}"
end
\end{verbatim}
The only difference between the for version of the loop and the eah version is
that the block of code that follows an iterator does define a new variable
scope.
\section{Iterators and Enumerable Objects}
Iterators example:
\begin{verbatim}
3.times(puts "haha"}
data.each {|x| puts x}
[1, 2, 3].map{|x| x*x }
factorial = 1;
2.upto(n) {|x| factorial *= x}
\end{verbatim}

The complex control structure behind this is \verb|yield|. The \verb|yield|
statement temporarily returns control from the iterator method to the method
that invoked the iterator.

When the end of the block is reached, the iterator method regains control and
execution resumes at the first statement following the yield.

Vertical bars at the start of a block are like parentheses in a method
definition. They hold a list of parameter names. The yield statement is like a
method invocation; it is followed by zero or more expressions whose values are
assigned to the block parameter.
\subsection{Numeric Iterators}
The Integer class defines three commonly used iterators:
\begin{verbatim}
4.upto(6) {|x| print x} # => pirnts "456"
3.times {|x| print x}   # => pirnts "012"
0.step(Math::PI, 0.1) { |x| puts Math.sin(x) }
\end{verbatim}
The last start form 0 and iterates in steps of 0.1 until it reaches Math::PI.
\subsection{Enumerable object}
\begin{verbatim}
[1,2,3].each { |x| print x } # prints "123"
(1..3).each { |print x}      # prints "123" same as 1.upto(3)

File.open(filename) do |f|      # open named file, pass as f
    f.each {|line| pirnt line } # print each line in f
end

squares = [1,2,3].collect {|x| x*x} # => [1,4,9]
evens = (1...10).selects {|x| x%2 == 0} # => [2,4,6,8,10]
odds = (1...10).reject {|x| x%2 == 0} # => [1,3,5,7,9]
\end{verbatim}
The inject method is more complicated. It invokes the associated block with two
arguments. The first argument is accumulated value of some sort from previous
iteration, the second argument is the next element of the enumerable object.The
return value of the block becomes the first block argument for the next
iteration. Or become the return value of the iterator after the last iteration.
\begin{verbatim}
data = [2,3,4,5]
sum = data.inject{|sum, x| sum + x } #=> 14
\end{verbatim}

\subsection{Writing Custom Iterators}
The defining feature of an iterator method is that it invokes a block of code
associated with the method invocation. Do this with the yield statement.

Invoking block twice:
\begin{verbatim}
def twice
    yield
    yield
end
\end{verbatim}
To pass argument values to the block, follow the yield statement with a
comma-separated list of expression.

The following example expects a block. It generate n values of the form
\verb|m*i + c| for i from 0 to n-1.
\begin{verbatim}
def sequence(n, m, c)
    i = 0
    while(i<n)
        yield m*i + c  #invoke the block, and pass the vlaue to it
    end
end

sequence(3, 5, 1) {|y| puts y}
\end{verbatim}
\subsection{Enumerators}
Of class \verb|Enumerable::Enumerator|. Usually use ts \verb|to_enum|. With no
arguments, \verb|to_enum| returns an enumerator whose each method simply calls
the each method of the target object. Call method with an enumerator instead of
a mutable array. This is a useful defensive strategy to avoid bugs.

You can also pass arguments The first argument should be a symbol that
identifies an iterator method.

The each method of the resulting Enumerator will invoke the named method of the
original object, any remaining arguments will be passed to that named method. 

\chapter{Altering Control Flow}
\section{Throw and Catch}
throw and catch are Kernel methods that define a control structure that can be
thought of as a multilevel break. Catch ca be in the calling method, or
somewhere even further up the call stack.

Ruby's catch method defines a labeled block of code, and Ruby's throw method
causes that block to exit.

Note that Ruby does exceptions differently, using raise and rescue. It is best
to consider throw and catch as a general-purpose control structure rather than
an exception mechanism. Example:
\begin{verbatim}
for matrix in data do
    catch: missing_data do
        for value in matrix do
            for value in row do
                throw:missing_data unless value
            end
        end
    end
end
\end{verbatim}
Note that the catch method takes a symbol argument and a block. It executes the
block and returns when the block exits or when the specified symbol is thrown.

If throw is never called, a catch invocation returns the value of the last
expression in its block.

If throw is called, then the return value of the corresponding catch is, by
default, nil. You can specify an arbitrary return value for catch by passing a
second argument to throw.

The return value can help you distinguish normal completion of the block from
abnormal completion with throw.

\section{Exception Handling}
Ruby uses Kernel method raise to raise exceptions, and use rescue clause to
handle exceptions. Exceptions raised by raise are instances of the Exception
class or one of its many subclass.

\chapter{Method, Procs, Lambdas, and Closures}t
Methods are defined with the \verb|def| keyword. This is followed by a method
name and an optional list of parameter names in parentheses. The ruby code that
consitutes the mehtod body follows the parameter list, and the end of the method
is marked with the \verb|end| keyword. 

Example:
\begin{verbatim}
# Define a factorial method with a single parameter n
def factorial(n)
    if n < 1
        raise "argument must be > 0"
    elsif n == 1
        1
    else
        n *factorial(n-1)
    end
end
\end{verbatim}
\section{Method Return Value}
Methods may terminate normally or abnormally.
\begin{description}
\item[Abnormal termination] Occurs when the method raises an eception.
\item[Normal termination] The value of the method invocation expression is the
value of the last expression is return.
\end{description}

The \verb|return| keyword is to force a return prior to the end of the method.
If an expression follows the \verb|return| keyword, then the value of that
expression is returned. If no expression follows, then the return value is
\verb|nil|. Eg.
\begin{verbatim}
def facotrial(n)
    raise "bad argument" if n < 1
    return 1 if n == 1
    n * factorial(n-1)
end
\end{verbatim}


Ruby method may return more than one value. To do this, use an explicit
\verb|return| statement and separate the values to be returned with commas:
\begin{verbatim}
#convert polar coordinates to Cartesian coordinates
def cartesian(magnitude, angle)
    [magnitude*Math.cos(angle), magnitude*Math.sin(angle) ]
end
\end{verbatim}
Methods of this form are typically intended for use with parallel assignment.

\subsection{Defining Singleton Method}
If we place a \verb|def| statement inside a \verb|class|, then the methods that
are defined are \emph{instance method of that method}, that is \textbf{these
methods are defined on all objects that are instance of the class}.

It is also possible to use \verb|def| to define a method on a \textbf{single specified
object} Simple \emph{follow the def keyword with an expression that
evaluates to an object}.

This expression should be followed by a period and the name of the method to be
defined:
\begin{verbatim}
o = "message"
def o.printme
    puts self
end
o.printme
\end{verbatim}
The resulting method is know as \emph{singleton method} because \textbf{it is
available only on a single object}.

\verb|Fixnum| and \verb|Symbol| cannot have singleton method because they are
immediate values.

Note:Methods may be undefined with the \verb|undef| statement.

\section{Method Names}
\begin{itemize}
\item By convention, method names begin with a lowercase letter, and use
underscore to separate it:\verb|like_this|
\item End with equal sign to be a \emph{setter}
\item Ends with question mark returns a value that answers the question posed by
the method invocation. Methods like these are \textbf{predicates}
\item Ends with an exclamation mark should be used with caution. Often they are
\emph{mutators} that do not end with an exclamation mark.
\end{itemize}


\section{Operator Methods}
Many of Ruby's operators, such as \verb|+,*,[]| are implemented with methods
that you can  define in your own class.

You define an operator by defining a method with the same "name" as the
operator.(Only exception are the unary plus and minus with \verb|@+| and
\verb|@-|

eg:
\begin{verbatim}
def +(other)    #Define binary plus operator: x+y is x.+(y)
    self.concatenate(other)
end
\end{verbatim}

Methods that define a unary operator are passed no arguments. Methods that
define binary operators are passed one argument and should operate on
\verb|self| and the argument.

The array access operator \verb|[]| and \verb|[]=| are special because they can
be invoked with \emph{any number of arguments}. For \verb|[]=|, the last
argument is always the value begin assigned.

\section{Method arguments}
\subsection{Hashes for named Arguments}
Write a method that expects a hash as its argument or as one of its argments:
\begin{verbatim}
# Generate sequence of n numbers. For any index i, 0 <=i < 0, 
# The value of element a[i] is m*i+c. Arguments n, m and c are
# Passed as keys in a hash, so that it is not necessary to remember
# their order
def sequence(args)
    # Extract the arguments from the hash.
    # Note the || operator to specify defaults used.
    # if the hash does not define a key that we are 
    #interested in
    n = args[:n] || 0
    m = args[:m] || 1
    c = args[:c] || 0

    a = []
    n.times {|i| a << m*i+c}
    a
end

sequence({:n=>3, :m=>5})
\end{verbatim}

In order to better support this style of programming, Ruby allows you to omit
the curly braces: \verb|sequnnce(:m=>3, :n=>5)|
\subsection{Block Arguments}
Block is a chunk of Ruby code associated with a method invocation, and that an
iterator is a method that expects a block.

One of the features of blocks is their anonymity They are not passed to the
method in a traditional sense, they are invoked with a keyword \verb|yield|
rather than with a method

If you prefer more explicit control over a block, add a final argument to your
method, and \emph{prefix the argument with an ampersand}. If you do this, then
that argument will be a \verb|Proc| object, and instead of using \verb|yield|,
you invoke that \verb|call| method:
\begin{verbatim}
def sequence(n, m, c, &b) 
    i = 0
    while(i < n) {
        b.call(i*m + c)
        i += 1
    end
end
\end{verbatim}

Notice that using the ampersand in this way changes only the method definition.
The method invocation remains the same.

Block arguments prefix with ampersands must be the last parameter.
\subsection{Using \& in the method invocation}
When \verb|&| is used before a \verb|Proc| object in a method invocation, it
treats the \verb|Proc| object as if it was an ordinary block following the
invocation.
\begin{verbatim}
a, b = [1,2,3], [4,5]
summation = Proc.new{|total, x| total + x} #A Proc object for summation}
sum = a.inject(0, &summation)
sum = b.inject(sum, &summation)
\end{verbatim}

In a method invocation an \verb|&| typically appears before a \verb|Proc|
object. But it is actually allowed before any object with \verb|to_proc| method.
The \verb|Method| class has such a method.

In 1.9, the \verb|Symbol| class define a \verb|to_proc| method, allowing symbols
to be prefixed with \verb|&| and passed to iterator:
\begin{verbatim}
words = ['and', 'but', 'car']
upper = words.map &:upcase
#equivelent to:
upper = words.map { |w| w.upcase}
\end{verbatim}

\section{Procs and Lambdas}
It is possible to create an object that represents a block. Depending on how the
object is created, it is called a \emph{proc} or \emph{lambda}

Procs are block-like behavior and lambdas have method-like behavior.Both,
however, are instances of \verb|Proc| class.

\subsection{Creating Procs}
Method that return a Proc:
\begin{verbatim}
def makeproc(&p)
    p
end
adder = makeproc {|x,y| x+y}
sum = adder.call(2,2)
\end{verbatim}

As well as creating procs by method invocation, there are three methods hat
create \verb|Proc| objects in ruby and it is not necessary to define a
\verb|makeproc| method.

\subsubsection{Proc.new}
It's the most obvious way to create a new instance of the \verb|Proc| class:
\begin{verbatim}
p = Proc.new{|x,y| x+y }
\end{verbatim}

\subsubsection{Kernel.lambda}
\verb|lmbda| is a method of the \verb|Kernel| module, so it behaves like a
global function. Lambda expects no arguments, but there must be a block
associated with the invocation:
\begin{verbatim}
is_positive = lambda {|x| x > 0}
\end{verbatim}

\subsubsection{Kernel.proc}
In 1.9, proc is a synonym for Proc.new

\subsubsection{Lambda Literals}
In Ruby 1.9, we can convert lambda declaraction to a literal as follows:
\begin{enumerate}
\item Replace the method name \verb|lambda| with the punctuation \verb|->|.
\item Move the list of arguments outside of and just before the curly braces
\item Change the argument list delimiters form || to ().
\end{enumerate}
\verb|succ = ->(x){ x+1 }|
One benefit of this new syntax is its succinctness. It can helpful when you want
to pass a lambda as an argument to a method or to another lambda.

Lambda literals create \verb|Proc| objects and are not the same thing as blocks

\subsubsection{Invoking Procs and Lambdas}
Procs and lambdas are objects, not methods, and they cannot be invoked in the
same way that methods are.

\verb|Proc| class defines a method named \verb|call|. Invoking this method
executes the code in the original block. The arguments you pass to the
\verb|call| method become arguments to the block, and return value value of the
block becomes the return value of the call method.

The \verb|Proc| class also defines the array access operator to work the same
way as \verb|call|. This means that you can invoke a proc or lambda using a
syntax that is like method invocation:
\begin{verbatim}
f = Proc.new { |x, y| 1.0/(1.0/x + 1.0/y) }
z = f.call(x,y)
z = f[x,y]
z = f.(x,y)    #Ruby 1.9 only
\end{verbatim}

\verb|.()| looks like a method invocation missing the method name. It can be
used with any object that defines a \verb|call| method and is not limited to
\verb|Proc| objects.

Ruby 1.9 adds a \verb|curry| method to the \verb|Proc| class. Calling this
method returns a curried version of proc or lambda. When a curried proc or
lambda is invoked with insufficient arguments it returns a new \verb|Proc|
object with the given arguments applied.
\begin{verbatim}
product = ->(x,y){x*y}
triple = product.curry[3]
\end{verbatim}

\section{Return in blocks, procs, and lambdas}
The \verb|return| statement in a block does not just return from the block to
the invoking iterator, \emph{it returns from the method that invoke the
iterator}.

A proc is like a block, so if you call a proc that executes a \verb|return|
statement, it attempts to return from the method that encloses the block that
was converted to the proc.

Lambdas work more like methods than blocks. A \verb|return| statement in a
lambda returns from the lambda itself, not from the method that surrounds the
creation site of the lambda.

\section{Break in blocks, procs and lambdas}
When we create a proc with \verb|Proc.new|, \verb|Proc.new| is the iterator that
\verb|break| would return from. And by the time we can invoke the proc object,
the iterator has already returned. So it never makes sense to have a top-level
\verb|break| statement.

If we create a proc object with an \verb|&| argument to the iterator method,
then we can invoke it and make the iterator return.

Lambdas are method-like, so putting a \verb|break| statement at the top-level of
a lambda, without an enclosing loop or iteration to break out of, doesn't
actually make any sense!

We might expect the following code to fail because there is nothing to break out
of in the lambda.
\section{Argument passing to procs and lambdas}
Invoking a block with \verb|yield| is similar to, but not the same as, invoking
a method:
\begin{itemize}
\item The \verb|yield| uses \emph{yield semantics},similar to parallel
assignment.
\item Method invocation uses \emph{invocation semantics}
\end{itemize}

Invoking proc uses yield semantics and invoking a lambda uses invocation
semantics. \verb|call| method.

\chapter{Class and Modules}
\section{Object Creation and Initialization}
\subsection{new, allocate, and initialize}
Every class inherits the class method new. This method has two jobs:
\begin{itemize}
\item It must allocate a new object
\item It must initialize the object.
\end{itemize}

If the new method were actually written in Ruby, it looks like this:
\begin{verbatim}
def new(*args)
    o = self.allocate
    o.initialize(*args)
end
\end{verbatim}

\verb|allocate| is an instance method of Class, and it is inherited by all class
objects. Its purpose is to create a new instance of the class.

\verb|initialize| is an instance method. Most classes need one, and every class
that extends a class other than Object should use super to chain to the
initialize method of the superclass. The usual job of the initialize method is
to create The return value of initialize is ignored. Ruby implicitly makes the
initialize method private.

\subsection{dup, clone, and initialize\_copy}
\verb|dup| allocate a new instance of the class of the object on which they are
invoked. They then copy all the instance variables and the taintedness of the
receiver object to the newly allocated object. Clone takes this copying a step
further than dup -- it also copies singleton methods of the receiver object and
freezes the copy object if the origin is frozen.

If a class defines a method named \verb|initialize_copy|, then the clone and dup
will invoke that method on the copied object after copying the instance
variables from the original.

When clone and dup copy instance variables from the original object to copy,
they copy references to the values of those variables; they do not copy the
actual values.

\subsection{marshal\_dump and marshal\_load}
A third way that objects are created is when \verb|Marshal.load| is called to
re-create objects previous marshaled with \verb|Marshal.dump|

\verb|Marshal.dump| saves the class of an object and recursively marshal the
value of each of its instance variables.

You can customize the way an object is marshaled by defining a
\verb|marshal_dump| instance method in the class; it should return a different
object (such as a string or an array of selected instance variable values) to be
marshaled in place of the receiver object.

If you define a custom \verb|marshal_dump|. It will be passed a reconstituted
copy of the object returned by \verb|marshal_dump|, and it must initialize the
state of the receiver object based on the state of the object it is passed.

\section{Modules}
Like a class, a module is a named group of methods, constants, and class
variables. Unlike class however, a module cannot be instantiated and it cannot
be subclassed. Modules
\end{document}

