\documentclass[11pt, a4paper]{book}
\usepackage{parskip}
\usepackage[top=1.5cm, left=1cm, right=1cm, bottom=2.5cm]{geometry}
\usepackage{fontspec}
\usepackage{graphicx}
\setmainfont{Roboto}
\setmainfont{DejaVu Serif}

\begin{document}
\chapter{Get Start}
\section{Hello, Rails!}
The \emph{Architecture of Rails Applications} is Model-View-Controller
framework.
\begin{enumerate}
\item Rails accepts incoming request from a browser
\item Decodes the request to find a controller.
\item Calls an action method in that controller
\item The controller invokes a particular view to display the results to the
user.
\end{enumerate}
For Hello world program, We need code for a controller and a view, and we need a route to connect the two.

Create Project:
\begin{verbatim}
rails new demo
\end{verbatim}

Create new controller:
\begin{verbatim}
rails generate controller Say hello goodbye
\end{verbatim}

The \verb|rails generate| command logs the files and directories it examines.

\subsection{Rails and Request URLs}
A rails application appears to its user to be associated with a URL. When you
point your browser at the URL, you are talking to the application code, which
generates a response to you.

\subsection{Our First Action}
We can see not only that have we connected the URL to our controller but also
that Rails is pointing the way to our next step, namely, to tell Rails what to
display.

That's where views comes in. The script to create controller generates six files
and a new directory to our application. That directory contains \emph{template
files for the controller's view}. 

In our case, we create a controller name say, so the views will be in he
directory \verb|app/views/say|

By default, \emph{Rails looks for templates in a file with the same name as the
action it's handling}.In our case, we need to replace \verb|hello.html.erb| in
the directory \verb|app/views/say|
\subsection{Making it dynamic}
\verb|ERb| is a filter that is installed as part of the Rails installation that
takes an \verb|.erb| file and outputs a transformed version.

Normal content is passed through without being changed. However, content between
\verb|<%=| and \verb|%>| is interpreted as Ruby code and executed. The result of
that execution is converted into a string, and that value is substituted in the
file in place of \verb|<%=...%>|.
\section{Linking Pages together}
Normally, each page in your application will correspond to a separate view.
We'll also use a new action method to handle the page. We'll use the same
controller for both actions.

Use \verb|<%= link_to "Goodbye", say_goodbye_path %>|. There is a \verb|link_to|
call within an ERb sequence. This creates a link to a URL that will invoke the
\verb|goodbye| action.
\begin{itemize}
\item The first parameter in the call is the text to be
displayed in the hyperlink,
\item And the next parameter tells rails to generate the
link to the goodbye action.
\end{itemize}

Stop for a minute to consider how we generate the link. We wrote this:
\begin{verbatim}
link_to "Goodbye!", say_goodbye_path
\end{verbatim}

\begin{itemize}
\item \verb|link_to| is a \emph{method call}. 
\item \verb|say_goodbye_path| is a precomputed value that Rails makes avaliable
to application views. It evaluates to the \verb|say/goodbye| path. 
\end{itemize}

\chapter{The Architecture of Rails Applications}
\section{Model, Views and Controllers}
\begin{description}
\item [Model] Maintaining the state of the application, more than just data,
\emph{it enforces all the business rules that apply to that data}. We put the
implementation of these business rules in the model and make sure that nothing
else in the application can make our data invalid. The model acts as
\textbf{gatekeeper and a data store}

\item [View] Responsible for generating a user interface, normally based on the
data in the model.The view itself \emph{never handles incoming data}, the view's
work is done once the data is displayed. 

\item [Controllers] orchestrate the application. Controllers receive events
from the outside world, interact with the model, and display an appropriate view
to the user. 
\end{description}

In a rails application:
\begin{enumerate}
\item Incoming request is first sent to a router, which \emph{works out where in
the application the request should be sent and how the request itself should be
parsed}
\item Routing identifies a particular method (called \emph{action}) in the
controller. The action might look at the data in the request, it might interact
with the model, and it might cause other application to be invoked.
\item Eventually the action prepares information for the view, which renders
something to the user.
\end{enumerate}
\section{Rails Model Support}
\subsection{Object-Relational Mapping}
ORM libraries map database tables to classes. If a database has a table called
\verb|orders|, our program will have a class named \verb|Order|. Rows in this
table correspond to objects of the class -- a particular order is represented as
an object of class \verb|Order|.
\begin{itemize}
\item Rows -- objects of the classes, a particular order is represented as an
object of class \verb|Order|, a particular order is represented as an object of
class \verb|Order|
\item Within that object, attributes are used to get and set the individual
columns.
\item The Rails classes that wrap our database tables provide a set of
class-level methods that perform table-level operations. For example, we might
need to find the order with a particular id. This is implemented as a class
method that returns the corresponding Order object.
\end{itemize}

So ORM layer maps tables to classes, rows to objects, and columns to attributes
of those objects.

Class methods are used to perform table-level operations.
\subsection{Active Record}
\emph{Active Record} is the ORM layer supplied with Rails. It differs from most
other ORM libraries in the way it is configured.

Here's a program that uses Active Record to wrap our orders table:
\begin{verbatim}
require 'active_record'

class Order < ActiveRecord::Base
end
order = Order.find(1)
order.pay_type = "Purchase order"
order.save
\end{verbatim}

This code uses the new \verb|Order| class to fetch the order with an id of 1 and
modify the \verb|pay_type|. (Creating connection to DB omitted)

\subsection{Action Pack: the View and Controller}
The controller supplies data to the view, and the controller receives events
from the pages generated by the views. Support for views and controllers in
Rails is bundled into a single component, \emph{Action Pack}

\subsubsection{View Support}
In Rails, the view is responsible for creating either all or part of a response
to be displayed in a browser, processed by templates, which come in three
flavors.

\begin{itemize}
\item Embedded Ruby(ERb), embeds snippets of Ruby code within a view document.
Flexible but we risk adding logic that should be in the model or controller.
\item XML Builder can also be used to construct XML document using Ruby code.
\item RJS views allow you to create JavaScript fragments on the server that are
then executed on the browser. This is great for creating dynamic Ajax interface.
\end{itemize}

\subsubsection{And the controller}
Home to a number of important ancillary services:
\begin{itemize}
\item Responsible for routing external requests to internal actions.
\item Manages caching.
\item Manages helper modules.
\item Manages sessions.
\end{itemize}

\chapter{Creating The application}
\section{Generating the Scaffold}
With Rails, you can do all that with one command by asking Rails to generate
what is know as \emph{scaffold} of given model.

Note that on the command line that follows, we use the singular form, Product.
In Rails, a model is \emph{automatically mapped to a database table whose name
is the plural form of hte model's class.} In our case, we asked for a model
called \verb|Product|, so Rails associate it with the table called
\verb|products|. And how will it find that table? The \verb|development| entry
in \verb|config/database.yml| tells Rails where to look for it. For SQLite3
users, this will be a file in db directory.)
\begin{verbatim}
depot> rails generate scaffold Product \
title:string description:text image_url:string price:decimal
\end{verbatim}
The generated file that we're interested in first is the migration one. A
migration \textbf{represents a change we want to make to the data}, expressed
in a source file in database-independent terms. These changes can update both
the database schema and the data in the database tables.

The migration has a UTC-based timestamp prefix, a name and a file extension.
\end{document}
\chapter{Router}
\section{Connecting URLs to Code}
Request: \verb|GET /patients/17|, it asks the router to match it to a controller
action, if the first matching route is:
\verb|match "patients/:id" => "patients#show"|, the request is dispatched to the
\verb|patients| controller's show action with \verb|{:id => "17"}| in
\verb|params|

\section{Generating Paths and URLs from Code}
You can also generate paths and URLs. If your application contains this code:
\begin{verbatim}
@patient = Patient.find(17)
...
<%= link_to "Patient Record", patient_path(@patient) %>
\end{verbatim}
The router will generate the path \verb|/patients/17|. This reduces the
brittleness of your view and makes your code easier to understand.

\section{Resource Routing: the Rails Default}
\subsection{Resources on the Web}
Browsers request pages from Rails by making a request for a URL using a specific
HTTP method. Each method is a request to perform an operation on the resource. A
resource route maps a number of related requests to action in a controller.

When receives:\verb|DELETE /photos/17|, it asks the router to map a controller
action. If the first matching route is:\verb|resources :photo|, Rails would
dispatch the request to the \verb|destroy| method on the \verb|photos|
controller with \verb|{:id=> "17"}| in \verb|params|

\verb|CRUD, Verbs and Actions|
In Rails, a resourceful route provides a mapping between HTTP verbs and URLs to
controller actions. By convention, each action also maps to particular CRUD
operations in a database. A single entry such as \verb|resources: photos|
creates seven different routes in your application, all mapping to the
\verb|PhotosController|

\subsection{Paths and URLs}
Creating a resourceful route will also expose a number of helpers to the
controllers in your application:
\begin{itemize}
\item \verb|photos_path| returns \verb|/photos|
\item \verb|new_photos_path| returns \verb|/photos/new|
\item \verb|edit_photo_path(:id)| returns \verb|/photos/:id/edit|
\item \verb|photo_path(:id)| returns \verb|/photo/:id|
\end{itemize}

Each of these helpers has a corresponding \verb|_url| helper (such as
\verb|photos_url|) which returns the same path prefixed with the current host,
port and path prefix.

\chapter{Controller}
\section{What does it do}
After routing has determined which controller to use for a request, your
controller is responsible for making sense of request and producing the
appropriate output.

For most conventional RESTful applications, the controller will receive the
request, fetch or save data from a model and use a view to create HTML output.
