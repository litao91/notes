\documentclass[11pt, a4paper]{book}
\usepackage{parskip}
\usepackage[top=1.5cm, left=1cm, right=1cm, bottom=2.5cm]{geometry}
\usepackage{fontspec}
\usepackage{graphicx}
\setmainfont{Roboto}
\setmainfont{DejaVu Serif}
\begin{Documents}
\chapter{Ajax in Rails}
\section{jQuery and the ajax() function}
Here is a quick refresher on how jQuery itself handles 

The special sauce that makes the asynchronous communication between the browser
and the server possible is the \verb|XMLHttpRequest|(XHL) object.

jQuery does this by means of the general utility funciton \verb|$.ajax()|.
The \verb|$.ajax()| function takes a single parameter, an \verb|options|
object. This object comprised of a set of key-value pairs for configuration. The
more commonly used:
\begin{enumerate}
\item \verb|url||: where to send the request to
\item \verb|type|: usually GET or POST
\item \verb|dataType|: the format of the data we expect back from the server.
Set the value of the "Accept" attribute in the HTTP request header object.
\item Callback functions to hook into various stages of the request, such
\verb|beforeSend|, \verb|complete|, \verb|error|, \verb|sucess|
\end{enumerate}
Also, \verb|$.get()|, \verb|$.getJcript()|, \verb|$.getJSON()|, \verb|$.post()|
All of these mehtods call the \verb|$.ajax()| in the backgrounnd.
\section{Rails and the jQuery UJS Adapter}
Before UJS Adapter:
\begin{verbatim}
//Adds a comment from upon click
$('a#add_comment').live('click', function(event) {
    $.ajax({
        url: $(this).href,
        dataType:'script'
    });
    return false;
})
\end{verbatim}

UJS adapter allows us to make exactly the same Ajax call with:

\end{Documents}
