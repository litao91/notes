\documentclass[a4paper, 11pt]{book}
\usepackage{parskip}
\usepackage[top=1.8cm, bottom=2.5cm, left=1.5cm, right=1.5cm]{geometry}
\usepackage{fontspec}
\setmainfont{DejaVu Serif}
\title{MVP of the Dophin Smalltalk}
\begin{document}
\chapter{Model-View-Cotroller}
In MVC, it is a view's responsibility to display the data held by a model
object. A controller can be used to determine how low-level user gestures are
translated into actions on the model.

Generally, a view and a controller are directly linked togehter (usually by an
instance variable pointing from one to the other) but a view/controller pair is
only indirectly linked to the model, meaning that an Oberserver relationship is
set up so that the view/controller pair knows about the existence of the model
but not vice versa.

Advantage: It is possible to connect multiple views/controllers pair to the same
model so that several user interfaces can share the same data.

In MVC, most of the application functionality must be built into a model class
know as an Application Model. It is then responsibility of the application model
to be the mediator between the true domain ojbects and the views and their
controllers.

The views are responsible for displaying the domain data while the controllers
handle the raw user gestures that will eventually perform actions on this data. 

So application model typically has methods to perform menu command actions, push
buttons actions and general validation on the data that it manages. Nearly all
of the application logic will reside in the application model classes.

For example, let's say one wants to explicitly change the color of one or more
views dependent on some conditions in the application model. The correct way to
do this in MVC would be to trigger some sort of event, passing the color along
with it. Behavior would then to be coded in the view to "hang off" this event
and to apply the color change whenever the event was triggered. 

\section{MVC In Smalltalk-80}
View manages the graphical and/or textural output to the portion of display

The controller interprets the mouse and keyboard inputs from the user,
commanding the model and/or the view to change as appropriate

The model manage the behavior and the data of the application domain, responds
to the requests for information about its state(usually from the view), and
 responds to instruction to change state (usually from the controller).

The formal separation of these three tasks is an important notion that is
particularly suited to Smalltalk-80 where the basic behavior can be embodied in
abstract objects: \emph{View}, \emph{Controller}, \emph{Model} and
\emph{Object}.

The MVC behavior is then inherited, added to, and modified as necessary to
provide a flexible and powerful system.

In smalltalk-80, input and output are largely stylized. Views must manage screen
real estate and display text or graphics from within that real estate.

Controllers must cooperate to ensure that the proper controller is interpreting
keyboard and mouse input (usually according to which view contains the cursor).
Because the input and output behavior of most application is sytlized, much of
it is inherited from the generic class View and Controller.

In contract, the model cannot be sytlized. Constraints on the type of objects 
allowed to function as models would limit the useful range of applictions 
possible within MVC paradigm.

\subsection{Communication Within The MVC Triad}
Communication between a view and its associated controller is straightforward
because View and Controller are specifically deisgned to work together. Models,
on the other hand, communicate in a more subtle manner.  Models, on the other 
hand, communicate in a more subtle manner.

\subsubsection{The Passive Model}
In the simplest case, it is not necessary for a model to make any provision
whatever for participation in a MVC triad. 

A simple WYSIWYG text editor is a good example:
The central property of such an editor is that you should always see the text as
it would appear on paper. So the view clearly must be informed of each change to
the text so that it can update its display. 

Yet the model (which we will assume is an instance of String) need not take 
responsibility for communicating the
changes to the view because these changes occur only by requests from the user.

The controller can assume responsibility for notifying the view of any changes
because it interprets the user's requests. It could simply notify the view that
something has changed -- the view could then request the current state of the
string from its model -- or the controller could specify to the view what has
changed. 

In either case, the string model is a completely passive holder of the
string data manipulated by the view and the controller. It adds, removes, or
replaces substrings upon demand from the controller and regurgitates appropriate
substrings upon request from the view. The model is totally "unaware" of the
existence of either the view or the controller and of its participation in an
MVC triad. That isolation is not an artifact of the simplicity of the model, but
of the fact that the model changes only at the behest of one of the other
members of the triad.

\subsection{The Model's Link to the Trial}
But all models cannot be so passive. Suppose that the data object -- the string
in the above example -- changes as a result of messages from objects other than
its view or controller. For instance, substrings could be appended to the end of
the string as is the case with the SystemTranscript. In that case the object
which depends upon the model's state -- its view -- must be notified that the
model has changed. Because only the model can track all changes to its state,
the model must have some communication link to the view.

 To fill this need, a
global mechanism in Object is provided to keep track of dependencies such as
those between a model and its view. This mechanism uses an IdentityDictionary
called \textbf{DependentFields} (a class variable of Object) which simply records all
existing dependencies. The keys in this dictionary are all the objects that have
registered dependencies; the value associated with each key is a list of the
objects which depend upon the key. In addition to this general mechanism, the
class Model provides a more efficient mechanism for managing dependents. When
you create new classes that are intended to function as active models in an MVC
triad, you should make them subclasses of Model. Models in this hierarchy retain
their dependents in an instance variable (dependents) which holds either nil, a
single dependent object, or an instance of DependentsCollection. Views rely on
these dependence mechanisms to notify them of changes in the model. When a new
view is given its model, it registers itself as a dependent of that model. When
the view is released, it removes itself as a dependent.

The methods that provide the indirect dependents communication link are in the
"updating" protocol of class Object. Open a browser and examine these methods.
The message "changed" initiates an announcement to all dependents of an object
that a change has occurred in that object. The receiver of the changed message
sends the message update: self to each of it's dependents. Thus a model may
notify any dependent views that it has changed by simply sending the message
self changed. The view (and any other objects that are registered as dependents
of the model) receives the message update: with the model object as the
argument. [Note: There is also a changed:with: message that allows you to pass a
parameter to the dependent.] The default method for the update: message, which
is inherited from Object, is to do nothing. But most views have protocol to
redisplay themselves upon receipt of an update: message. This changed/update
mechanism was chosen as the communication channel through which views can be
notified of changes within their model because it places the fewest constraints
upon the structure of models.
\end{document}

