\documentclass[a4paper,11pt]{article}
\usepackage{parskip}
\topmargin -1.5cm
\oddsidemargin -0.04cm
\evensidemargin -0.04cm
\textwidth 16.59cm
\textheight 25cm

\author{LI Tao}
\title{Metapost Notes}
\begin{document}
\section{Basic Drawing Statements}
\begin{tabbing}
Drawing a Line:\hspace{1cm}\= \verb|draw (20,20)--(0,0)|\\
Draws a diagnal line:\>\verb|draw (20,20)--(0,0)--(0,30)--(30,0)--(0,0)|\\
Draw a single point:\>\verb|drawdot (30,0)|\\
Own scale factor, say $u$:\> $u=1cm$\\
Settign line width:\>\verb|pickup pencircle scaled 4pt|\\
\end{tabbing}
The default coordinate system:for coordinates like $(30,0)$, it means that it is 30 units to the right of the origin, where a unit is $\frac{1}{72}$ inch.
\paragraph{Examples}
\begin{verbatim}
for i=0 upto 2:
    for j=0 upto2: drawdot(i*u,j*u); endfor
endfor
\end{verbatim}
The result is a three-by-three grid of bold dots.
\section{The MetaPost Workflow}
A stand-alone MetaPost compiler:\verb|mpost|.

The executable is actual a small wrapper program around \emph{mplib}, a library containing the MetaPost compiler.

Creating graphics with MetaPost follows the \emph{edit-compile-debug} paradigm.

To create a graphics with MetaPost, you prepare a text file containing code in the MetaPost language and then invoke the compiler, usually by giving a command of the form:\verb|mpost <input file>|

The file usually have names ending \verb|.mp|

Any terminal I/O during compilation, MetaPost ouptputs \verb|<job-name>.log|

If all goes well during compilation, MetaPost outputs one or more graphics files in a variant of the PostScript format, by defaut.

A MetaPost input file normally containts a sequence of \verb|beginfig()| , \verb|endfig| pairs with an end statement\footnote{Omitting the final end statement causes to enver interactive mode after processing the input files}. These are macros that perform various administrative functions and ensure that the result of all drawing operations get packaged up and translated into PostScript or SVG format.

Assignments to internal variables:
\begin{verbatim}
prologus:=3;
outputtemplate:="%j-%c.mps";
outputformat:="svg";
\end{verbatim}
If your graphics contain text labels, you might want to set variable proloues to 3 to make sure the correct fonts are used under all possible circustances.

The second assignment changes the output file name scheme to the form \verb|<jobname>-<n>.mps|.

\section{Curves}
A \verb|draw| statement with the points separated by \verb|..| draws a smooth curve through the points.

Just use \verb|--|where you want straight lines and \verb|..| where you want curves.

\subsection{Bezier Cubic Curves}
When MetaPost asked to draw a smooth curve through a sequence of points, it constructes a piecewise curve with continuous slop and approximately continuous curvature. 
\section{Linear Equation}
For example, the MetaPost interpreter can read \verb|a+b=3; 2a=b+3;| and deduce that a=2 and b=1.
\section{Expressions}
It is possible to experiment with expressions by involving any of the data types metioned below by using the statement \verb|show<expression>|

\subsection{Data Types}
Ten basic data types:numeric, pair, path, transform, (rbg)color, cmykcolor, string, boolean, picture, and pen.

Numeric quantities in MetaPost are represented in fixed point arithmetic as integer multiples of $\frac{1}{65536}$
\end{document}
