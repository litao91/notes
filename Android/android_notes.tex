\documentclass[11pt, a4paper]{book}
\usepackage{parskip}
\usepackage[top=1.5cm, left=1cm, right=1cm, bottom=2.5cm]{geometry}
\usepackage{fontspec}
\setmainfont{DejaVu Serif}
\begin{document}
\chapter{Classes and Methods}
\section{Scroller}
\subsection{Class Overview}
This class encapsulates scrolling. The duration of the scroll can be passed
 in the constructor and specifies the maximum time that the scrolling animation
should take. Past this time, the scrolling is automatically moved to its final
stage and \verb|computeScrollOffset()| will always return false to indicate taht
scrolling is over.

\section{AppWdiget}

\subsection{AppWidgetManager}
Updates AppWidgetState, gets information about installed AppWidgetProviders and
other Widget related state.
\subsection{AppWidgetHost}
Provides the \emph{interaction} with the AppWidget source for apps, like the
home screen, that want to embed AppWidget in their UI.

You need to privde a hostid at construction, the hostId is a number of your 
choosing that should be internally unique to your app (that is, you don't 
need to worry about collisions with other apps on the system).  It's designed 
for cases where you want two unique AppWidgetHosts inside of the same 
application, so the system can optimize and only send updates to actively 
listening hosts. 

\subsection{AppWidgetHostView}
Provides the glue to show AppWidgetViews. Offers automatic animation between
updates, and will try recycling old views for each incoming.
\subsection{AppWidgetProviderInfo}
Describes the meta data for an installed AppWidgetProvider. The fields
correspond to the fields in the \verb|appwidget-provider>| xml tag.


\section{Content Providers}
\subsection{Overview}
Content providers manage access to a \emph{structured set of data}. They
encapsulate the data, and provide mechanisms for defining data security. Content
providers are the standard interface that \emph{connets data in one process with
code running in another process.}

Use \verb|ContentResolver| Object in application's Context to communicate with
the provider as a client. 

\verb|ContentResolver| object communicates with the provider object, an instance
of a class that implements \emph{ContentProvider}. The provider object receives
data requests from clients. 

A content provider manages access to a central repository of data. A provider is
part of an Android application, which ofen privides its own UI for working with
the data. 

However, content providers are primarily intended to be \emph{used by other
applications}, which access the provider using a provider client object. 

\subsection{Content URIs}
A content URI is a URI that indentifies data in a provider. Content URIs include
the symbolic name of the entire provider (its authority) and name that points to
a table (a path).
\section{Handler}
A Handler allows you to send and process Message and Runnable objects associated
with a thread's MessageQueue

Each instance is aassociated with a single thread and that thread's message
queue.

When you create a Handler, it is bound to the thread/message queue of the thread
that is creating it -- from that point on, it will deliver messages and
runnables to that message queue and execute them as they come out of the message
queue. 

Two main uses:
\begin{enumerate}
\item To schedule messages and runnable to be executed as some point in the
future;
\item To enqueue an action to be performed on a different thread than your own.
\end{enumerate}

The \verb|post(Runnable)| allow you to enqueue Runnable objects to be called by
the messager queue when they are received;

The \verb|sendMessage(Message)| allow you to enqueue a Message object
containning a bundle of data that will be processed by the Handler's
\verb|handleMessage(Messsage)| method.

When posting or sending to a Hnadler, you can either allow the item to be
processed as soon as the message queue is ready to do so, or specify a dely. 
\begin{description}
\item[notifyChange] Notify registered obsevers that a row was updated. 
\end{description}
\section{ViewGroup}
\subsection{onInterceptTouchEvent}
Implement this method to intercept all touch screen \verb|MotionEvent|. This
allow you to \emph{watch event as they are dispatched to your children}, and
take the owner ship of the current gesture at any point.

Events will be received in the following order:
\begin{enumerate}
\item You will receive the \textbf{down event} in \verb|onInterceptTouchEvent|
\item The \textbf{down event} will be handled either by a child of this view
group, or given to the \verb|ViewGroup|'s own \verb|onTouchEvent()| method to
handle. This means you should implement \verb|onTouchEvent()| to return
\verb|true|, so you will continue to see the rest of the gesture (instead of
looking for a parent view to handle it). Also by returning true from
\verb|onTouchEvent|, you will not receive any following events in
\verb|onInterceptTouchEvent()| and all touch processing must happen in
\verb|onTouchEvent()|
\item For as long as you return \verb|false| from \verb|onInterceptTouchEvent|,
each following event will be delivered first here and then to the target's
\verb|onTouchEvent()|
\item If you return \verb|true| here, you will not receive any following events,
the target view will receieve the same event but with the \verb|ACTION_CANCEL|
and all further events will be delivered to your \verb|onTouchEvent()| method
and no longer appear here.
\end{enumerate}

\chapter{View Animation}
You can use the view animation system to perform tweened animation on Views.
Tween animation calculated the animation with information such as the start
point, end point, size, rotation, and other common aspect of an animation.

A tween animation can perform a series of simple transformations on the contents
of a View object. A sequence of animation instructions defines the tween
animation, defined by either XML or Android code.  

The animation instructions define the transformation that you want to occur,
when they will occur, and how long they should take to apply. Transformations ca
be sequential or simultaneous.

Each transformation takes a set of parameters specific for that transformation,
and also a set of common parameters.

To make several transformation happen simultaneously, give them the same stat
time; to make them sequential, calculate the start time plus the duration of the
preceding transformation.

\section{Defining in XML}
The animation XML file belongs in the \verb|res/anim/| directory of your Android
project. The file must have a single root element:this will be either a single
\verb|<alpha>|, \verb|<scale>|, \verb|<translate>|, \verb|<rotate>|,
interpolator element, or \verb|<set>|element that hold groups of these
elements.To make them occur sequentially, you must specify the
\verb|startOffset| attribute.

You can determine \emph{how a transformation is applied over time} by assigning
an \verb|Interpolator|.

With XML saved as \verb|hyperspace_jump.xml| in the \verb|res/anim/| directory
of the project, the following code will reference it and apply it to an
\verb|ImageView| object from the layout
\begin{verbatim}
ImageView spaceshipImage = (ImageView) findViewById(R.id.spaceshipImage);
Animation hyperspaceJumpAnimation = AnimationUtils.loadAnimation(this, R.anim.hyperspace_jump);
spaceshipImage.startAnimation(hyperspaceJumpAnimation);
\end{verbatim}

XML examples:
\begin{verbatim}
<?xml version="1.0" encoding="utf-8"?>

<alpha xmlns:android="http://schemas.android.com/apk/res/android"
       android:interpolator="@android:anim/accelerate_interpolator"
       android:fromAlpha="0.0" android:toAlpha="1.0" android:duration="100" />
\end{verbatim}
\begin{verbatim}
<?xml version="1.0" encoding="utf-8"?>

<layoutAnimation xmlns:android="http://schemas.android.com/apk/res/android"
        android:delay="10%"
        android:order="reverse"
        android:animation="@anim/slide_right" />
\end{verbatim}
\section{Defining in Java code}
Example of sliding a view down from the top:
\begin{verbatim}
 AnimationSet set = new AnimationSet(true);

  Animation animation = new AlphaAnimation(0.0f, 1.0f);
  animation.setDuration(100);
  set.addAnimation(animation);

  animation = new TranslateAnimation(
      Animation.RELATIVE_TO_SELF, 0.0f, Animation.RELATIVE_TO_SELF, 0.0f,
      Animation.RELATIVE_TO_SELF, -1.0f, Animation.RELATIVE_TO_SELF, 0.0f
  );
  animation.setDuration(500);
  set.addAnimation(animation);

  LayoutAnimationController controller =
      new LayoutAnimationController(set, 0.25f);
\end{verbatim}
Notes on this code:
\begin{itemize}
\item The animation ssequence is defined in Java, as an AnimationSet object, to
which various Animation subclasses can be added.
\item You have to create \verb|LayoutAnimationController| which will actually
orchestrate the sequence/AnimationSet that you've defined .
\end{itemize}
\section{Applying animation sequences}
Once animation sequences are defined in XML or java, they can be applied to
Views or ViewGroups and run.
\subsection{Layout animation}
When applying a layout animation to a ViewGroup, you \emph{don't have to start
or stop the animation sequence}. You can't pause it. When you add or remove a
View from your ViewGroup, the animation sequence you have specified will run at
that moment.
\subsubsection{Loading layout animation from Java}
\begin{verbatim}
public static void setLayoutAnim_slidedownfromtop(ViewGroup panel, Context ctx) {

  AnimationSet set = new AnimationSet(true);

  Animation animation = new AlphaAnimation(0.0f, 1.0f);
  animation.setDuration(100);
  set.addAnimation(animation);

  animation = new TranslateAnimation(
      Animation.RELATIVE_TO_SELF, 0.0f, Animation.RELATIVE_TO_SELF, 0.0f,
      Animation.RELATIVE_TO_SELF, -1.0f, Animation.RELATIVE_TO_SELF, 0.0f
  );
  animation.setDuration(500);
  set.addAnimation(animation);

  LayoutAnimationController controller =
      new LayoutAnimationController(set, 0.25f);
  panel.setLayoutAnimation(controller);
}
\end{verbatim}
The LayoutAnimationController is used by the ViewGroup, who's layout is being
animated, to determine how your AnimationSet will be orchestrated and drawn.

Finally, once the layout controller has been created, after the animation set is
defined, you have to bind it to a ViewGroup that will automatically invoke this
animation controller, which will run the set, when the layout is changed.

\subsubsection{Loading layout animation from XML}
\begin{verbatim}
public static void setLayoutAnimation2(ViewGroup panel, Context ctx) {

  LayoutAnimationController controller = AnimationUtils.loadLayoutAnimation(ctx, R.anim.app_enter);

  panel.setLayoutAnimation(controller);

}
\end{verbatim}
\subsection{View animation}


\chapter{About XML and Layout}
\section{Include to Reduce}
A component can be seen as a complex widget made of several simple stock
widgets. Creating new components can be done easily by writing a custom
\verb|View| but it can be done even more easily using only XML.

In Android XML layout file, each tag is mapped to an actual class instance. The
UI toolkit lets you also use three special tags that are not mapped to a 
\verb|View| instance:\verb|<requestFocus />|, \verb|<merge />| and
\verb|<incude\>|. The latter \verb|<include/>| can be used to create pure XML
visual components.

The \verb|<include />| includes another XML layout. Using this tag is straight
forwad as shown in the follwing example:
\begin{verbatim}
<com.android.launcher.Workspace
    android:id="@+id/workspace"
    android:layout_width="fill_parent"
    android:layout_height="fill_parent"

    launcher:defaultScreen="1">

    <include android:id="@+id/cell1" layout="@layout/workspace_screen" />
    <include android:id="@+id/cell2" layout="@layout/workspace_screen" />
    <include android:id="@+id/cell3" layout="@layout/workspace_screen" />

</com.android.launcher.Workspace>
\end{verbatim}

In the \verb|<include />| only the layout attribute is required. This attribute
\emph{without the android namespace prefix, is a reference to the layout file
you wish to include}. You can override all the layout parameters.



\section{Optimize by merging}
The \verb|<merge/>| was created for the purpose of optimizing Android layouts by
reducing the number of levels in view trees. 

Example: The following XML layout declares a layout that shows an image with its
title on top of it:
\begin{verbatim}
<FrameLayout xmlns:android="http://schemas.android.com/apk/res/android"
    android:layout_width="fill_parent"
    android:layout_height="fill_parent">

    <ImageView  
        android:layout_width="fill_parent" 
        android:layout_height="fill_parent" 
    
        android:scaleType="center"
        android:src="@drawable/golden_gate" />
    
    <TextView
        android:layout_width="wrap_content" 
        android:layout_height="wrap_content" 
        android:layout_marginBottom="20dip"
        android:layout_gravity="center_horizontal|bottom"

        android:padding="12dip"
        
        android:background="#AA000000"
        android:textColor="#ffffffff"
        
        android:text="Golden Gate" />

</FrameLayout>
\end{verbatim}
Since our FrameLayout has the same dimension as its parent, by the virtue of
using the \verb|fill_parent| constraints, and does not define any background,
extra padding or a gravity, it is \emph{totally useless}. We only made the UI
more complex. Since XML documents require a root tags in XML layout always
represent view instance, the \verb|<merge />| tag comes in handy.

When the LayoutInflater encounters this tag, it skips it and adds the
\verb|<merge />| children to the \verb|<merge />| parent.

So in this example, it's better to replace FrameLayout with merge:
\begin{verbatim}
<merge xmlns:android="http://schemas.android.com/apk/res/android">

    <ImageView  
        android:layout_width="fill_parent" 
        android:layout_height="fill_parent" 
    
        android:scaleType="center"
        android:src="@drawable/golden_gate" />
    
    <TextView
        android:layout_width="wrap_content" 
        android:layout_height="wrap_content" 
        android:layout_marginBottom="20dip"
        android:layout_gravity="center_horizontal|bottom"

        android:padding="12dip"
        
        android:background="#AA000000"
        android:textColor="#ffffffff"
        
        android:text="Golden Gate" />

</merge>
\end{verbatim}

So both the \verb|textView| and the \verb|ImageView| will be added directly to
the top-level \verb|FrameLayout|. The result will be \emph{visually the same but
the view hierarchy is simpler}

The \verb|<merge />| can be useful in other situations. For instance, it works
perfectly when combined with the \verb|<include />| tag. You can also use
\verb|<merge />| when you create a custom composite view.


\end{document}
